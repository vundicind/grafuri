\title{Curs 10: Rețele; Fluxuri}

\begin{document}

\maketitle

\begin{frame}
  \frametitle{Rețea}

O \emph{rețea} este un graf orientat cu două vîrfuri speciale: un vîrf fără predecesori (vîrf sursă), al doilea fără succesori (vîrf destinație); iar fiecare arc are asociată o anumită \emph{capacitate}.

Formal o rețea este o structură $N=(G,s,t,c)$, unde:
\begin{itemize}
  \item $G$ - digraf;
  \item $s$ și $t$ - sursa și destinația;
  \item $c:E(G)\to\mathbb{R}$ - funcția \emph{capacitate}.
\end{itemize}

\end{frame}

\begin{frame}
  \frametitle{Flux în rețea}

Fiind dat un digraf $G$ și o funcție $f:E(G)\to\mathbb{R}$ vom nota
\[
  f(X,Y)=\sum_{e\in E(X,Y)} f(e) \text{ pentru orice } X,Y\subseteq V(G).
\]

În particular, $f(x,Y) = f(\{x\},Y\setminus \{x\})$.

Un \emph{flux} într-o rețea $N=(G,s,t,c)$ este o funcție $f:E(G)\to\mathbb{R}$ care verifică condițiile:
\begin{enumerate}
  \item[(F1)] $f(u,v)=-f(v,u)$ pentru orice $(u,v)\in E(G)$, $u\neq v$;
  \item[(F2)] $f(v,V(G))=0$ pentru orice $v\in V(G)\setminus\{s,t\}$;
  \item[(F3)] $f(e)\leq c(e)$ pentru orice $e\in E(G)$.
\end{enumerate}

\end{frame}

\begin{frame}
  \frametitle{Flux în rețea}

\begin{figure}
\centering%
\begin{tikzpicture}
  \SetVertexMath

  \Vertex{s}
  \SetGraphUnit{2}
  \NOEA[NoLabel](s){v1}
  \SOEA[NoLabel](s){v2}
  \SetGraphUnit{3}
  \EA[NoLabel](v1){v3}
  \EA[NoLabel](v2){v4}
  \SetGraphUnit{2}
  \NOEA(v4){t}

  \SetUpEdge[style={->}]
  \Edges[label=3](s,v1) \Edges[label=0](s,v2)
  \Edges[label=1](v1,v3) \Edges[label=1](v1,t) \Edges[label=3](v1,v4)
  \Edges[label=2](v2,v1) \Edges[label=2](v4,v2)
  \Edges[label=1](v3,t)
  \Edges[label=1](v4,t)
\end{tikzpicture}
\caption{Diestele,p142; Flux în rețea}
\end{figure}

\end{frame}

\begin{frame}
  \frametitle{Flux în rețea}

Evident, din $(F1)$ pentru orice $X\subseteq V(G)$ avem $f(X,X)=0$.

Un flux se numește \emph{integral} dacă toate valorile acestuia sînt numere întregi.

Valoarea $f(v,V(G))$ se numește \emph{fluxul net} în vîrful $v$. 

\end{frame}

\begin{frame}
  \frametitle{Tăietură}

O \emph{tăietură} în rețea este perechea $(S,V(G)\setminus S)$ unde $S\subseteq V(G)$ cu $s\in S$ și $t\in V(G)\setminus S$.

\begin{theorem}
Orice tăietură $(S,V(G)\setminus S)$ într-o rețea verifică 
\[
  f(S,V(G)\setminus S)=f(s,V(G)).
\]
 
\end{theorem}


\end{frame}

\begin{frame}
  \frametitle{Tăietură}

\begin{proof}
\[
  \begin{array}{ll}
    f(S,S^c)	&= f(S,V) - f(S,S)\\
    (F1)	&= (f(s,V) + f(S\setminus\{s\},V)) - 0\\
    (F2)	&= f(s,V)
  \end{array}
\]
 
\end{proof}

Valoarea comună a $f(S,S^c)$ se numește \emph{valoare} fluxului $f$ și se notează $|f|$.

\end{frame}

\begin{frame}
  \frametitle{Ford și Fulkerson}

Din (F3) avem că $f(S,S^c)\leq c(S,S^c)$; respectiv valoarea totală nu întrece cea mai mică capacitate a unei tăieturi.

Următoarea teoremă spune că acestă margine este mereu atinsă.

\begin{theorem}[Ford și Fulkerson]
În orice rețea, maximul valorii totale a unui flux este egală cu capacitatea minimă a tăieturilor.
\end{theorem}

\end{frame}

\begin{frame}
  \frametitle{Circulații; Grupuri}

Fiind dat un semigrup abelain cu 0, $H$, o \emph{$H$-circulație} într-o rețea este o funcție $f:E(G)\to H$ încît:
\begin{itemize}
  \item[(F1)] $f(u,v)=-f(v,u)$ pentru orice $(u,v)\in E(G)$, $u\neq v$;
  \item[(F2')] $f(v,V(G))=0$ pentru orice $v\in V(G)$.
\end{itemize}

 
\end{frame}

\begin{frame}
  \frametitle{Circulații; Grupuri}

Evident, din $(F1)$ pentru orice $X\subseteq V(G)$ avem $f(X,X)=0$.

În plus, din $(F2')$ , 
\[
  f(X,V(G))=\sum_{x\in X}f(x,V)=0.
\]

\begin{theorem}
Dacă $X\subseteq V$ atunci $f(X,X^c)=0$.
\end{theorem}

\end{frame}


\begin{frame}
  \frametitle{Circulații; Grupuri}

\begin{proof}
\[
  f(X,X^c)=f(X,V)-f(X,X)=0-0=0.
\]

\end{proof}

\begin{corollary}
Dacă $f$ este o circulație și $e$ o punte atunci $f(e)=0$. 
\end{corollary}


\end{frame}

\begin{frame}
  \frametitle{Circulații; Grupuri}

Dacă $H$ este un grup abelian atunci toate circulațiile pe $G$ formează un grup într-un mod natural cu operațiile

$(f+g)(e)=f(e)+g(e)$ și $(-f)(e)=-f(e)$.

O circulație cu proprietatea $f(e)\neq 0$ pentru orice $e\in E$ se numește $H$-flux.

$H$-fluxurile nu formează un grup. 

Din ultimul corolar un graf care are un $H$-flux nu poate avea punți.

\end{frame}

\begin{frame}
  \frametitle{Circulații; Grupuri}

Un $\mathbb{Z}$-flux se numește $k$-flux, $k\in\mathbb{N^*}$, dacă $0<|f(e)|<k$ pentru orice $e\in E$.

Evident un $k$-flux este $l$-flux pentru orice $l>k$.

Astfel apare întrebarea: care este cel mai mic $k$ îcît $G$ să admită $k$-flux?

Acest număr se notează $\varphi(G)$ si se numește \emph{număr de flux}.

Dacă $G$ nu are $k$-fluxuri pentru orice $k$ punem $\varphi(G)=\infty$.

\end{frame}

\begin{frame}
  \frametitle{$k$ și $\mathbb{Z}_k$}

\begin{theorem}
Un multigraf admite $k$-flux dacă și  numai dacă admite $\mathbb{Z}_k$-flux.  
\end{theorem}

\end{frame}


\begin{frame}
  \frametitle{$k$ și $\mathbb{Z}_k$}

\begin{theorem}
Un graf are un 2-flux dacă și numai dacă toate vîrfurile au grad par.  
\end{theorem}

\begin{proof}
Graful $G=(V,E)$ are 2-flux dacă și numai dacă are un $\mathbb{Z}_2$-flux.

Fie $f:E\to\mathbb{Z}_2$, $f(xy)=1$.

Funcția $f$ verifică proprietatea $(F1)$ deoarece $f(yx)=1=-1$.

Funcția $f$ va verifica proprietatea $(F2')$ numai dacă fiecare vîrf va avea grad par. 
\end{proof}


\end{frame}

\begin{frame}
  \frametitle{$k$ și $\mathbb{Z}_k$}

\begin{theorem}
Pentru orice $n>4$, $\varphi(K_n)=3$. 
\end{theorem}

\end{frame}

\begin{frame}
  \frametitle{Fluxuri și colorări}

Fie $G=(V,E)$ un graf planar și $G^*=(V^*,E^*)$ dualul său.

Fie $g$ o circulație pe $G^*$ ce poate induce acestă circulație pe $G$?

De exemplu cunoaștem că tăieturile minime pe $G^*$ sînt cicluri în $G$.

Astfel din proprietatea $g(X,X^c)=0$ avem că suma $g$ pentru toate muchiile unui ciclu în $G$ trebuie să fie zero.

\end{frame}

\begin{frame}
  \frametitle{Fluxuri și colorări}

\begin{theorem}
Fie $G$ digraf plan și $G^*$ digraful său dual. Atunci există o bijecție de la $E$ la $E^*$ cu proprietățile
\begin{enumerate}
  \item ???
\end{enumerate}

\end{theorem}

\end{frame}

\begin{frame}
  \frametitle{Fluxuri și colorări}

\begin{theorem}[Tutte]
Pentru orice pereche $G$ și $G^*$ de graf plane și dualul său $\chi(G)=\varphi(G^*)$. 
\end{theorem}
\end{frame}

\end{document}

