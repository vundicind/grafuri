\usepackage[utf8x]{inputenc}
\usepackage[romanian]{babel}
%\usepackage[T2A]{fontenc}
\usepackage{amsmath}
\usepackage{amsfonts}
\usepackage{amssymb}
\usepackage{hyperref}
\usepackage{ntheorem}
  \theoremstyle{change}
  \theorembodyfont{\upshape}
  \newtheorem{theorem}{Teoremă}[section]
  \newtheorem{definition}[theorem]{Definiție}
  \newtheorem{example}[theorem]{Exemplu}
  \newtheorem{proposition}[theorem]{Propoziție}
  \newtheorem{corollary}[theorem]{Corolar}
  \newenvironment{proof}{{\bf Demonstrație:} }{}
  %\newtheorem{exercise}[theorem]{Exercițiu}
  \newtheorem{question}[theorem]{\^{I}ntrebare}
  \newtheorem{problem}[theorem]{Problemă}
  \newtheorem{algorithm}[theorem]{Algoritm}
\usepackage{verbatim}
\usepackage{graphicx}
\usepackage{hyperref}
\usepackage{color}
\usepackage{caption}
\usepackage{subcaption}
\usepackage{titlesec}
  \titleformat{\chapter}
    {\normalfont\huge\bfseries}{}{20pt}{\Huge}
  \titlespacing*{\chapter}{0pt}{50pt}{40pt}
  \titleclass{\section}{straight}
\usepackage{xargs}
\usepackage{minibox}% http://tex.stackexchange.com/questions/8680/how-can-i-insert-a-newline-in-a-framebox
  \renewcommandx\minibox[3][1=l, 2=c]{%
    \begin{tabular}[#2]{@{}#1@{}}
      #3
    \end{tabular}%
  }