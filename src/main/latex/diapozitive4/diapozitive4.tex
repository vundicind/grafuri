
\title{Curs 4: Arbori}

\begin{document}

\maketitle

\begin{frame}
  \frametitle{Arbori}

Un \emph{arbore} este un graf conex fără cicluri.\pause

Vîrfurile de gradul 1, într-un arbore, se numesc \emph{frunze}.\pause

Arborele care constă doar dintr-un vîrf se numește arbore \emph{trivial}; adică $K_1$.\pause

Orice arbore netrivial are cel puțin o frunză.

\end{frame}

\begin{frame}
  \frametitle{Arbori}

\begin{figure}
\centering%
\begin{tikzpicture}
  \SetVertexNoLabel

  \Vertex{v}
  \EA(v){x1} \NO(x1){y1} \SO(x1){z1}
  \EA(x1){x2} \NO(x2){y2} \SO(x2){z2}
  \EA(x2){x3} \NO(x3){y3} \SO(x3){z3}
  \EA(x3){x4}
  \Edges(v,x1,x2,x3,x4)
  \Edges(y1,x1,z1) \Edges(y2,x2,z2) \Edges(y3,x3,z3)

  \begin{scope}[shift={(7,0)}]
    \grStar[RA=1]{6}
  \end{scope}

  \begin{scope}[shift={(0,-3)}]
    \grPath[RA=1]{4}
  \end{scope}

  \begin{scope}[shift={(7,-3)}]
    \Vertex{v}
    \SOWE(v){x1}
    \SOEA(v){x2}
      \SOEA(x2){x3}
      \SOWE(x2){x4}
    \Edges(v,x1,v,x2,x3,x2,x4)
  \end{scope}

\end{tikzpicture}
\end{figure}

\end{frame}

\begin{frame}
  \frametitle{Arbori; Punți}

Conexitatea arborelui este una slabă; este de ajuns să suprimăm o muchie sau un vîrf care nu-i frunză din arbore și graful obținut nu mai este conex.

Din acest punct de vedere arborii sînt grafuri minimal conexe.\pause

\begin{theorem}
Un graf conex $G$ este arbore dacă și numai dacă orice muchie a sa este punte.
\end{theorem}
\end{frame}


\begin{frame}
  \frametitle{Arbori; Punți}

\begin{proof}
{\em Suficiența}. Fie $G$ conex și orice muchie e punte. Atunci nici o muchie 
nu poate să se conțină într-un ciclu. \pause

Deci este aciclic.\pause

{\em Suficiența}. Fie $G$ arbore, deoarece nu conține cicluri reiese că fiecare 
muchie este punte.
\end{proof}

\end{frame}



\begin{frame}
  \frametitle{Lanț unic}

\begin{theorem} 
Într-un arbore pentru oricare două vîrfuri distincte, există un unic lanț elementar care le unește.
\end{theorem}\pause

\begin{proof} 
Fie $G=(V,E)$ un arbore. \pause

Presupunem, prin absurd, că există două vîrfuri diferite $u$ și $v$ pentru care $G$ conține cel puțin două $u-v$-lanțuri elementare distincte.\pause

Notăm aceste lanțuri prin $P_1$ și $P_2$.\pause

Întrucît $P_1\neq P_2$ există cel puțin un vîrf $x$ care aparține $P_1$ și nu aparține $P_2$.\pause

Dar atunci există o muchie $e=xy$ în $P_1$ care nu aparține $P_2$. 

\end{proof}

\end{frame}


\begin{frame}
  \frametitle{Lanț unic}

\begin{proof}[Demonstrație; Continuare]

Din cele de mai sus reiese că subgraful $(P_1\cup P_2)-e$ este conex; adică conține un $x-y$-lanț. \pause

Dar în așa caz subgraful $(P_1\cup P_2)+e$ va conține un ciclu, ceea ce este imposibil deoarece $G$ este arbore. 
\end{proof}

\end{frame}

\begin{frame}
  \frametitle{Lanț unic}

\begin{corollary}
Un graf este arbore dacă și numai dacă pentru oricare două vîrfuri distincte, 
există un unic lanț elementar care le unește.
\end{corollary}\pause

\begin{proof}
Evident, dacă orice două vîrfuri sînt unite printr-un unic lanț reiese că $G$ 
este în I rînd conex și în al II-lea rînd aciclic.
\end{proof}
 
\end{frame}

\begin{frame}
  \frametitle{Numărul de vîrfuri vs. numărul de muchii}

 \begin{theorem}\label{GTA-2.1}
Dacă $G$ este un arbore atunci $\|G\|=|G|-1$.
\end{theorem}\pause

\begin{proof}
Demonstrăm prin inducție tare pe $n=|G|$ (numărul de vîrfuri).\pause

\emph{Pasul I:} Pentru $n=1$ avem graful trivial la care numărul de muchii este 0. \pause

Deci teorema este verificată.\pause

\emph{Pasul I:} Presupunem că teorema este adevărată pentru orice arbore cu numărul de vîrfuri mai mic decît $n$. \pause

Fie $G$ un arbore cu $n$ vîrfuri, $n\geq 2$. \pause

Alegem o muchie $e=uv\in E(G)$; atunci unicul $u-v$-lanț este însăși muchia $e$.


\end{proof}

\end{frame}


\begin{frame}
  \frametitle{Numărul de vîrfuri vs. numărul de muchii}

\begin{proof}[Demonstrație; Continuare]
Atunci $G-e$ este neconex și $G-e$ constă exact din două componente; fiecare dintre acest două componente fiind arbore. 

\begin{figure}
\centering%
\begin{tikzpicture}
  %\draw[help lines] (0,0) grid (5,5);

  \SetGraphUnit{3}
  \Vertex[x=1,y=1]{C} 
  \Vertex[x=2,y=2]{D}
  \Vertex[x=4,y=2]{E}

  \Vertex[x=4.7,y=0.2]{J}
  \Vertex[x=3,y=0.7]{I}
  \Vertex[x=4.8,y=1.2]{H}
  \Vertex[x=0.4,y=2.6]{B}
  \Vertex[x=4.6,y=4]{G}
  \Vertex[x=3,y=3.3]{F}
  \Vertex[x=1,y=4.3]{A}

  \Edges(B,A,F,E,G)
  \Edges(C,D,I,J)
  \Edge(E)(H)
  \only<1>{\Edge(E)(I)}
  
  \only<3->{\node (G1) at ([yshift=5pt]$ (B.center) !.4! (F.center) $) {$G_1$};}  
  \only<3->{\node (G2) at ([yshift=-5pt]$ (C.center) !.4! (I.center) $) {$G_2$};}  
  
  \begin{scope}[on background layer]
    \only<3->{\fill [blue!20,draw] plot [smooth cycle] coordinates {%
          ([xshift=-10pt] B.south west) (A.north) (F.north) %
          ([yshift=10pt] G.north) (H.south east) %
          ($ (B.center) !.6! (F.center) $)};}
    \only<3->{\fill<3-> [red!20,draw] plot [smooth cycle] coordinates {%
      ([xshift=-10pt] C.south west) ([yshift=10pt] D.north) %
      ([yshift=10pt] I.north) ([yshift=5pt] J.north) %
      ([xshift=10pt] J.east) ([yshift=-10pt] J.south) %
      ([yshift=-10pt] I.south)};}
  \end{scope}

\end{tikzpicture} 
\end{figure}

\only<3->{Notăm prin $G_1$ și $G_2$ acesși doi arbori. }

\end{proof}
 
\end{frame}

\begin{frame}
  \frametitle{Numărul de vîrfuri vs. numărul de muchii}

\begin{proof}[Demonstrație; Continuare]
Evident, $|G_1|,|G_2|<n$; dar atunci

\begin{equation}
  \|G_1\|=|G_1|-1 \label{GTA-2.1-1}
\end{equation}

și
\begin{equation}
  \|G_2\|=|G_2|-1. \label{GTA-2.1-2}
\end{equation}\pause

Aplicînd \eqref{GTA-2.1-1} și \eqref{GTA-2.1-2} obținem
\[
\begin{array}{rl}
  \|G\| & =\|G_1\| + \|G_2\| + 1\\
        & =\left(|G_1| - 1\right) + \left(|G_2| - 1\right) + 1\\
        & =|G_1|+|G_2|-1=|G|-1.
  \end{array}
\]
 
\end{proof}

\end{frame}

\begin{frame}
  \frametitle{Numărul de vîrfuri vs. numărul de muchii}

 \begin{corollary}
Orice arbore netrivial are cel puțin două vîrfuri cu gradul 1.
\end{corollary}\pause

\begin{proof}
Fie $G=(V,E)$ un $(n, m)$-arbore netrivial atunci

\[
\begin{array}{rl}
  \sum_{v\in V}d(v) & = 2m\\
                    & = 2(n-1)\\
                    & = 2n-2.
\end{array}  
\]\pause

Pe de altă parte $G$ este netrivial și deci $d(v)\geq 1$, $\forall v\in V$.\pause

Reiese că $G$ trebuie să aibă cel puțin două vîrfuri cu gradul 1.
\end{proof}
\end{frame}

\begin{frame}
  \frametitle{Arbori ``scufundate'' în alte grafuri}

\begin{theorem}
Fie $G$ un arbore cu $m$ muchii și $H$ un graf cu $\delta(H)\geq m$. Atunci $H$ are un subgraf izomorf cu $G$.
\end{theorem}

\end{frame}

\begin{frame}
  \frametitle{Arbore de acoperire}

\begin{definition}
  Un subgraf de acoperire conex și fără cicluri se numește \emph{arbore de acoperire}. 
\end{definition}

Doar un graf conex poate avea arbore de acoperire.

\end{frame}

\begin{frame}
  \frametitle{Arbori de acoperire}

\begin{figure}
\centering%
\begin{tikzpicture}
  \SetVertexMath

  \mygrHouse

  \begin{scope}[shift={(4,0)}]
    \SetUpEdge[color=gray!10]
    \mygrHouse
    \Edges[color=black](z,x,u,v,y)
  \end{scope}

  \begin{scope}[shift={(8,0)}]
    \SetUpEdge[color=gray!10]
    \mygrHouse
    \Edges[color=black](v,u,v,x,z,x,v,y)
  \end{scope}

\end{tikzpicture}
\caption{De la stînga spre dreapta: un graf împreună cu doi arbori de acoperire ai săi}
\end{figure}

\end{frame}

\begin{frame}
  \frametitle{Aplicații ale arborilor de acoperire}

???

\end{frame}



\begin{frame}
  \frametitle{Arbori de acoperire}

\begin{corollary}
Orice graf conex $G$ conține un arbore de acoperire.
\end{corollary}
\begin{proof}
Fie $H$ un subgraf minimal de acoperire al lui $G$ care este conex. 

Atunci pentru orice muchie $e$ din $H$, $H-e$ nu mai este conex. 

Reiese că orice muchie a lui $H$ este punte, concludem că $H$ este un arbore.
\end{proof}
\end{frame}

\begin{frame}
  \frametitle{Arbori de acoperire}


\begin{corollary}
Dacă $G$ este conex atunci $\|G\|\geq |G|-1$.
\end{corollary}
\begin{proof}
Fie $H$ un arbore de acoperie al lui $G$ atunci pe de o parte $\|H\|=|H|-1$, 
iar pe de altă parte $\|G\|\geq \|H\|$ și $|G|=|H|$.
\end{proof}
 
\end{frame}


\begin{frame}
  \frametitle{Arbori de acoperire}

\begin{corollary}
Un graf conex $G$ este arbore dacă și numai dacă $\|G\|=|G|-1$
\end{corollary}
\begin{proof}
Necesitatea a fost deja demonstrată. 

{\em Suficiența}. Presupunem, prin absurd, că $G$ nu este arbore. 

Atunci pentru două vîrfuri $u$ și $v$ putem găsi două $uv$-lanțuri elementare. 

Reiese că $G$ conține un ciclu $C$. 

Fie $e$ o muchie din $C$ atunci $G-e$ rămîne conex. 

Dar atunci $\|G-e\|=\|G\|-1$ și $\|G-e\|=n-2$ adică $G-e$ nu mai este conex; Contradicție.
\end{proof}
 
\end{frame}

\begin{frame}
  \frametitle{Numărul de arbori de acoperire}

În general, după cum s-a observat, un graf conex poate avea mai mult de un arbore de acoperire.

Vom nota prin $\tau(G)$ numărul de arbori de acoperire ale grafului $G$. 

Evident, dacă $G$ este arbore atunci $\tau(G)=1$.
\end{frame}

\begin{frame}
  \frametitle{Numărul de arbori de acoperire}

\begin{theorem}
Dacă $G$ este un graf conex atunci
\[
  \tau(G)=\tau(G-e)+\tau(G/e)	\quad	\forall e\in E(G).
\]

\end{theorem}

Acestă teoremă poate fi utilizată pentru a calcula, într-un mod recursiv, numărul de arbori de acoperire a unui graf conex.

\end{frame}

\begin{frame}
  \frametitle{Aplicații}

\begin{figure}
\centering%
\begin{tikzpicture}
  \SetVertexMath

  \begin{scope}
    \grCycle[RA=1,prefix=v]{4}
    \Edge[style={bend left}](v1)(v3)
    \Edge[style={bend right}](v1)(v3)
    \draw (0,-1.5) node (G) {$G$};
  \end{scope}

  \begin{scope}[shift={(3.5,0)}]
    \grEmptyCycle[RA=1,prefix=v]{4}
    \Edges(v1,v2,v3,v0)
    \Edge[style={bend left}](v1)(v3)
    \Edge[style={bend right}](v1)(v3)
    \draw (0,-1.5) node (G1) {$G_1=G-v_0v_1$};
  \end{scope}

  \begin{scope}[shift={(6.5,0)}]
    \Vertex[L=v_4,x=0,y=1]{v4}
    \Vertex[L=v_2,x=-1,y=0]{v2}
    \Vertex[L=v_3,x=0,y=-1]{v3}
    \Edge[style={bend left}](v4)(v3)
    \Edge[style={bend right}](v4)(v3)
    \Edge[style={bend right,out=60,in=120}](v4)(v3)
    \Edges(v4,v2,v3)
    \draw (0,-1.5) node (G2) {$G_2=G/v_0v_1$};
  \end{scope}

\end{tikzpicture}
\end{figure}

\begin{figure}
\centering%
\begin{tikzpicture}
  \SetVertexMath

  \begin{scope}
    \grEmptyCycle[RA=1,prefix=v]{4}
    \Edges(v1,v2,v3,v0)
    \Edge[style={bend left}](v1)(v3)
    \Edge[style={bend right}](v1)(v3)
    \draw (0,-1.5) node (G1) {$G_1$};
  \end{scope}

  \begin{scope}[shift={(3.5,0)}]
    \grEmptyCycle[RA=1,prefix=v]{4}
    \Edges(v2,v3,v0)
    \Edge[style={bend left}](v1)(v3)
    \Edge[style={bend right}](v1)(v3)
    \draw (0,-1.5) node (G1) {$G_3=G_1-v_1v_2$};
  \end{scope}

  \begin{scope}[shift={(6.5,0)}]
    \Vertex[L=v_0,x=1,y=0]{v0}
    \Vertex[L=v_4,x=0,y=1]{v4}
    \Vertex[L=v_3,x=0,y=-1]{v3}
    \Edge[style={bend left}](v4)(v3)
    \Edge[style={bend right}](v4)(v3)
    \Edge[style={bend left,out=300,in=240}](v4)(v3)
    \Edges(v3,v0)
    \draw (0,-1.5) node (G2) {$G_4=G_1/v_1v_2$};
  \end{scope}

\end{tikzpicture}
\end{figure}

\end{frame}

\begin{frame}

\begin{figure}
\centering%
\begin{tikzpicture}
  \SetVertexMath

  \begin{scope}
    \Vertex[L=v_4,x=0,y=1]{v4}
    \Vertex[L=v_2,x=-1,y=0]{v2}
    \Vertex[L=v_3,x=0,y=-1]{v3}
    \Edge[style={bend left}](v4)(v3)
    \Edge[style={bend right}](v4)(v3)
    \Edge[style={bend right,out=60,in=120}](v4)(v3)
    \Edges(v4,v2,v3)
    \draw (0,-1.5) node (G2) {$G_2$};
  \end{scope}

  \begin{scope}[shift={(3.5,0)}]
    \Vertex[L=v_4,x=0,y=1]{v4}
    \Vertex[L=v_2,x=-1,y=0]{v2}
    \Vertex[L=v_3,x=0,y=-1]{v3}
    \Edge[style={bend left}](v4)(v3)
    \Edge[style={bend right}](v4)(v3)
    \Edge[style={bend right,out=60,in=120}](v4)(v3)
    \Edges(v2,v3)
    \draw (0,-1.5) node (G2) {$G_5=G-v_2v_4$};
  \end{scope}

  \begin{scope}[shift={(6.5,0)}]
    \Vertex[L=v_4,x=0,y=1]{v5}
    \Vertex[L=v_3,x=0,y=-1]{v3}
    \Edge[style={bend left}](v5)(v3)
    \Edge[style={bend right}](v5)(v3)
    \Edge[style={bend right,out=60,in=120}](v5)(v3)
    \Edge[style={bend left,out=300,in=240}](v5)(v3)
    \draw (0,-1.5) node (G2) {$G_6=G/v_2v_4$};
  \end{scope}

\end{tikzpicture}
\end{figure}

\[
  \begin{array}{ll}
    \tau(G)	&= \tau(G_1)+\tau(G_2)\\
		&= (\tau(G_3)+\tau(G_4))+(\tau(G_5)+\tau(G_6))\\
		&= (2+3)+(3+4)\\
		&= 12.
  \end{array}
\]

 
\end{frame}





\end{document}

