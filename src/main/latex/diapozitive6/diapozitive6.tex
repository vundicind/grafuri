
\title{Curs 6: Cuplaje; Factorizări}


\begin{document}

\maketitle

\begin{frame}
  \frametitle{Definiția cuplajului}

Un \emph{cuplaj} al unui graf $G$ este o submulțime $M\subseteq E(G)$ astfel încît orice vîrf al lui $G$ este incident cu cel mult o singură muchie din $M$.\pause

\begin{figure}
\centering%
\begin{tikzpicture}
  \SetVertexNoLabel

  \mygrLadder

  \begin{scope}[shift={(3,0)}]
    \mygrLadderMattchingA
  \end{scope}

  \begin{scope}[shift={(6,0)}]
    \mygrLadderMattchingMaximal
  \end{scope}

  \begin{scope}[shift={(9,0)}]
    \mygrLadderMattchingMaxim
  \end{scope}
  
\end{tikzpicture}
\end{figure} \pause

Un cuplaj este o submulțime de muchii neadiacente două cîte două (adică \emph{mulțime independentă de muchii}).

\end{frame}

\begin{frame}
  \frametitle{Vîrfuri saturate}

Dacă un vîrf $v$ este incident cu o muchie din cuplajul $M$ spunem că:

\begin{itemize}
  \item vîrful $v$ este \emph{cuplat} (sau \emph{saturat}) de cuplajul $M$;
  \item\mbox{} [sau că] $M$ \emph{cuplează} (sau \emph{saturează}) vîrful $v$.
\end{itemize}

\begin{figure}
\centering%
\begin{tikzpicture}
  \SetVertexMath

  \mygrLadder

  \begin{scope}[shift={(3,0)}]
    \mygrLadderMattchingA
  \end{scope}
 
\end{tikzpicture}
\end{figure}


\end{frame}

\begin{frame}
  \frametitle{Mulțimi saturate}

Putem extinde noțiunea de ``vîrf saturat'' la noțiunea de ``mulțime saturată''.\pause

Dacă $U\subseteq V(G)$ este o submulțime care conține în exclusivitate vîrfuri saturate de cuplajul $M$, atunci spunem că:\pause

\begin{itemize}
  \item mulțimea $U$ este \emph{cuplată} (sau \emph{saturată}) de cuplajul $M$;\pause
  \item\mbox{} [sau că] $M$ \emph{cuplează} (sau \emph{saturează}) mulțimea $U$.
\end{itemize}
 

\end{frame}


\begin{frame}
  \frametitle{Cazuri particulare de cuplaje}

\begin{itemize}[<+->]
  \item \emph{Cuplaj perfect} - cuplajul care saturează toate vîrfurile grafului.
  \item \emph{Cuplaj maximal} - cuplajul care este subgraf maximal în raport cu proprietatea de a fi cuplaj;
    \begin{itemize}
      \item dacă $M$ este un cuplaj maximal atunci $\forall e\in E(G)\setminus E(M)$, $M+e$ nu mai este cuplaj. 
    \end{itemize}

  \item \emph{Cuplaj maxim} - cuplajul de cardinalitate maximă;
    \begin{itemize}
      \item cuplajul maxim este cuplajul cu cel mai mare numar de muchii în raport cu toate celelalte cuplaje ale grafului;
      \item pot fi mai multe cuplaje maxime, dar cardinalul acestora este un număr unic;
      \item cardinalul cuplajelor maxime se numește \emph{număr de muchie-independență} și se noteaza $\nu(G)$.
    \end{itemize}
\end{itemize}


\end{frame}


\begin{frame}
  \frametitle{Cazuri particulare de cuplaje}

\begin{figure}
\centering%
\begin{tikzpicture}
  \SetVertexNoLabel

  \mygrLadder

  \begin{scope}[shift={(3,0)}]
    \mygrLadderMattchingA
  \end{scope}

  \begin{scope}[shift={(6,0)}]
    \mygrLadderMattchingMaximal
  \end{scope}

  \begin{scope}[shift={(9,0)}]
    \mygrLadderMattchingMaxim
  \end{scope}
  
\end{tikzpicture}
\end{figure}

Ultimul cuplaj este maxim și perfect; Penultimul cuplaj este maximal.

Orice cuplaj maxim este cuplaj maximal? Dar invers?

Orice cuplaj perfecte este cuplaj maximal sau maxim? 

Orice cuplaj maxim este cuplaj perfect?
 
\end{frame}

\begin{frame}
  \frametitle{Problema cuplajului maxim}

\emph{Fiind dat un graf să se determine un cuplaj maxim.}

În cele ce urmează ne vom ocupa de condițiile necesare și suficiente pentru ca un cuplaj să fie maxim. 

\end{frame}

\begin{frame}
  \frametitle{Lanț alternat}

Un lanț elementar $P$ se numește \emph{alternant relativ la cuplajul} $M$ dacă muchiile lui $P$ aparțin alternativ mulțimilor $M$ și $E(G)\setminus M$.\pause

Un lanț alternant relativ la cuplajul $M$ se numește \emph{de creștere relativ la cuplajul} $M$ dacă extremitățile acestui lanț sînt distincte și  nu sînt saturate de $M$.\pause

Prima și ultima muchie a unui lanț de creștere relativ la un careva cuplaj nu aparțin acestui cuplaj. 

\end{frame}

\begin{frame}
  \frametitle{Lanț alternat}

\begin{figure}
\centering%
\begin{tikzpicture}
  \SetVertexMath

  \mygrLadder

  \begin{scope}[shift={(3,0)}]
    \mygrLadderMattchingA
  \end{scope}

\end{tikzpicture}
\end{figure}\pause

Lanțul $(v_1,u_2,v_0,u_0)$ este un lanț de creștere relativ la cuplajul indicat (dreapta).\pause

Lanțul $(u_2,v_0,u_0)$ este un lanț alternant, dar nu este un lanț de creștere.
 
\end{frame}

\begin{frame}
  \frametitle{Cuplaje maxime}

\begin{theorem}[Berge]
Un cuplaj $M$ este cuplaj maxim în graful simplu $G$ dacă și numai dacă nu există în $G$ lanțuri de creștere relativ la $M$. 
\end{theorem}\pause

Ideea teoremei rezidă în faptul că dacă $P$ este un lanț de creștere relativ la cuplajul $M$ atunci mulțimea $M\Delta E(P)$ este un cuplaj ce are cu o muchie mai mult decît $M$.\pause 

Așadar condițiile ``un cuplaj $M$ este maxim'' și ``există un lanț de creștere relativ la $M$'' se exclud mutual.

\end{frame}

\begin{frame}
  \frametitle{Acoperirea muchiilor cu vîrfuri}
 
De la condiții care se exclud mutual trecem la condiții care sînt duale.\pause

O \emph{acoperiere a muchiilor cu vîrfuri} (sau \emph{suport} al grafului, sau \emph{transversală}) în $G$ este o submulțime $U\subseteq V(G)$ astfel încît orice muchie $e\in E(G)$ este incidentă cu cel puțin un vîrf din $U$.

\end{frame}

\begin{frame}
  \frametitle{Acoperirea muchiilor cu vîrfuri}

\begin{figure}
\centering%
\begin{tikzpicture}
  \SetVertexMath

  \mygrLadder

  \begin{scope}[shift={(3,0)}]
    \mygrLadderCovering

  \end{scope}

\end{tikzpicture}
\caption{ De la stînga spre dreapta: un graf împreună cu o acoperire a muchiilor}
\end{figure}

\end{frame}


\begin{frame}
  \frametitle{Cuplaje maxime}

Evident, pentru orice graf $G$ mulțimea $V(G)$ este o acoperire a muchiilor maximă.\pause

Din acest motiv are sens să căutăm mulțimi de acoperire a muchiilor minime.\pause

\begin{theorem}[K\" onig]
Fie $G$ un graf bipartit atunci numărul de muchie-independență este egal cu cardinalul mulțimilor de acoperire a muchiilor minime.
\end{theorem}

\end{frame}




\begin{frame}
  \frametitle{Cuplaje perfecte}

Nu orice graf conține cuplaje perfecte; de exemplu: 

\begin{figure}
\centering%
\begin{tikzpicture}
  \SetVertexNoLabel
  \grEmptyCycle[RA=1,rotation=45]{4}
  \grComplete[x=3,RA=1]{3}
  \grComplete[x=6,RA=1,rotation=7.2]{5}
\end{tikzpicture}
  
\end{figure}


\alert{Care-s condițiile necesare și suficiente pentru ca un graf să conțină un cuplaj perfect?}

Pentru a găsi răspuns la acestă întrebare:
\begin{enumerate}
  \item vom cerceta grafurile bipartite;
  \item vom cerceta grafurile generale, dar migrînd la alt unghi de vedere.
\end{enumerate}



\end{frame}

\begin{frame}
  \frametitle{Grafuri bipartite}

\alert{Cînd un graf bipartit are un cuplaj perfect?}\pause

Fie $G$ un graf bipartit cu bipartiția mulțimii vîrfurilor $\{X,Y\}$. \pause

Evident, un cuplaj perfect trebuie să satureze vîrfurile din $X$.\pause

Evident, pentru ca să existe un cuplaj care să satureze $X$ trebuie ca fie care vîrf din $X$ să aibă suficienți vecini în $Y$, adică  
$|N(S)|\geq |S|$, $S\subseteq X$.



\end{frame}


\begin{frame}
  \frametitle{Grafuri bipartite}

Cînd un graf bipartit are un cuplaj perfect?\pause

Fie $G$ un graf bipartit cu bipartiția $\{X,Y\}$. Un cuplaj perfect saturează vîrfurile din $X$.\pause

Pentru ca să existe un cuplaj care să satureze $X$ trebuie ca fie care vîrf din $X$ să aibă suficienți vecini în $Y$, 
$|N(S)|\geq |S|$, $S\subseteq X$.\pause

\begin{theorem}[Hall]
Un graf bipartit $G$ cu $\{X,Y\}$ conține un cuplaj pentru $X$ dacă și numai dacă  $|N(S)|\geq |S|$, $S\subseteq X$
\end{theorem}\pause

\begin{corollary}[Teorema Căsătoriei]
Dacă $G$ este bipartit $k$-regulat, $k\geq 1$, atunci $G$ conține un cuplaj perfect. 
\end{corollary}


\end{frame}


\begin{frame}
  \frametitle{Problema căsătoriei}

Dacă fiecare fată dintr-un sat cunoaşte exact $p$ băieţi şi fiecare băiat cunoaşte exact $q$ fete atunci fiecare fată se poate mărită cu un băiat pe care-l cunoaşte şi fiecare băiat poate lua de soţie o fată pe care o cunoaşte dacă și numai dacă $p=q$.

\end{frame}

\begin{frame}
  \frametitle{Demonstrația teoremei căsătoriei}

Dacă $G$ este $k$-regulat rezultă că $|X|=|Y|=k$.\pause

Așadar (în baza teoremi Hall) este suficient să arătăm că există un cuplaj care saturează mulțimea $X$.
Pentru în teorema lui Hall se ține cont doar de gradele vîrfurilor.\pause

Fie $S\subseteq X$ are $k|S|$ vecini.\pause

Astfel $|N(S)|=k|S|\geq |S|$.

\end{frame}


\begin{frame}
  \frametitle{Grafuri generale cu alt unghi de vedere: noțiunea de $k$-factor}

\begin{itemize}[<+->]
  \item Orice subgraf de acoperire $k$-regulat al unui graf $G$ se numește \emph{$k$-factor} [al lui $G$];
    \begin{itemize}
      \item în particular, \alert{1-factor = cuplaj perfect}.
    \end{itemize}
  \item Mai general, orice subgraf de acoperire al unui graf $G$ se numește \emph{factor} [al lui $G$];
    \begin{itemize}
      \item o \emph{factorizare} a grafului $G$ este o mulțime de factori muchie-disjuncți reuniuena cărora este $G$;
      \item în particular, \emph{1-factorizare} - mulțime de 1-factori muchie-disjuncți a căror reunine este $G$.
    \end{itemize}
  \item În matematică, cuvîntul ``factor'', deseori apare în cazurile cînd elementele ``similare'' ale unei mulțimi/clase sînt grupate în raport cu o careva relație de echivalență;
    \begin{itemize}
      \item grupul factor $\mathbb{Z}_n$ al grupului $\mathbb{Z}$.
    \end{itemize}
\end{itemize}


\end{frame}

\begin{frame}
  \frametitle{Exemplu de $k$-factori și $k$-factorizare}

%\begin{minipage}{0.5\textwidth}
\begin{figure}
\centering%
\begin{tikzpicture}
  \SetVertexNoLabel
  \grComplete[RA=1,rotation=45]{4}
    \draw (0,-1.2) node (G){$G$};

  \begin{scope}[shift={(3,0)}]
    \grEmptyCycle[RA=1,rotation=45]{4}
      \Edge(a0)(a1); \Edge(a2)(a3)
      \draw (0,-1.2) node (H1){$H_1$};
  \end{scope}

  \begin{scope}[shift={(6,0)}]
    \grEmptyCycle[RA=1,rotation=45]{4}
      \Edge(a0)(a3); \Edge(a1)(a2)
      \draw (0,-1.2) node (H2){$H_2$};
  \end{scope}

  \begin{scope}[shift={(0,-3)}]
    \grEmptyCycle[RA=1,rotation=45]{4}
      \Edge(a0)(a2); \Edge(a1)(a3)
      \draw (0,-1.2) node (H3){$H_3$};
  \end{scope}

  \begin{scope}[shift={(3,-3)}]
    \grEmptyCycle[RA=1,rotation=45]{4}
      \draw (0,-1.2) node (H4){$H_4$};
  \end{scope}    

  \begin{scope}[shift={(6,-3)}]
    \grEmptyCycle[RA=1,rotation=45]{4}
      \Edge(a1)(a2); \Edge(a2)(a3); \Edge(a0)(a2)
      \draw (0,-1.2) node (H5){$H_5$};
  \end{scope}    

\end{tikzpicture}%
\caption{Un graf $G$ împreună cu subgrafurile sale: $H_1$,$H_2$, ...,$H_5$.
Subgrafurile $H_1$,$H_2$,$H_3$ sînt 1-factori și îmreună formează o 1-factorizare;
$H_4$ este o 0-factorizare; $H_5$ este factor, dar nu și $k$-factor.
}
\end{figure}
%\end{minipage}

\end{frame}

\begin{frame}
  \frametitle{O condiție necesară de existență a 1-factorilor}

Să ne întoarcem la întrebarea din care cauză am introdus noțiunea de ``1-factor''.\pause

Și anume: \alert{care-s condițiile necesare și suficiente pentru ca într-un graf să existe 1-factori}.\pause

\alert{Pentru moment o condiție necesară:}\pause

\begin{itemize}[<+->]
  \item Dacă un graf conține 1-factori atunci acesta are un număr par de vîrfuri;
    \begin{itemize}
      \item un 1-factor nu poate să existe pe un număr impar de vîrfuri;
      \item într-un graf numărul de vîrfuri de grad impar trebuie să fie par.
    \end{itemize}
\end{itemize}


\end{frame}

\begin{frame}
  \frametitle{Încă o condiție necesară de existență a 1-factorilor}

Notăm, pentru orice graf $G$, prin $q(G)$ numărul de componente de ordin impar (adică, care au un numar impar de vîrfuri).\pause

\alert{Încă o condiție necesară:}\pause

Dacă $G$ conține 1-factori atunci $q(G\setminus S)\leq |U|$ pentru orice $U\subseteq V(G)$.

\end{frame}

\begin{frame}
  \frametitle{Demonstrație}

\begin{enumerate}[<+->]
  \item Fie $F$ un 1-factor al grafului $G$.
  \item Fie $U\setminus V(G)$ și presupunem că $q(G-U)=k$.
  \item Alegem o numerotare oarecare pentru componentele impare: $H_1, H_2, ..., H_k$.
  \item Considerăm $H_i$; $H_i\cap F$ nu poate fi un 1-factor pe $H_i$ întrucît acesta din urmă are un numar impar de vîrfuri.
  \item Așadar în $H_i$ trebuie să existe un vîrf $x_i$ care să fie unit/cuplat prin $F$ cu un vîrf $y_i\notin H_i$.
  \item Vîrful $y_i\in U$;
    \begin{itemize}
      \item dacă îndepărtarea mulțimii $U$ a dat naștere la componentele $H_1,H_2,...,H_k$ reiese că în graful inițial nu existau muchii între vîrfurile din $H_i$ și $H_j$ (toate lanțurile treceau prin $U$).
    \end{itemize}
  \item Așadar $q(G\setminus U)\leq |U|$.
\end{enumerate}


\end{frame}

\begin{frame}
  \frametitle{O condiție necesară și suficientă de existență a 1-factorilor}

\begin{theorem}[Tutte]
Un graf $G$ conține un 1-factor dacă  $q(G\setminus S)\leq |U|$ pentru orice $U\subseteq V(G)$. 
\end{theorem}\pause

\begin{corollary}[Petersen]
Orice graf cubic (3-regulat) fără punți conține un 1-factor. 
\end{corollary}




\end{frame}




\end{document}

