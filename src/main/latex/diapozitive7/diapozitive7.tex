\title{Curs 7: Colorare}

\begin{document}

\maketitle

\begin{frame}
 \frametitle{Colorare}

\begin{definition}
Fie $G=(V,E)$ un graf. O \emph{$k$-colorare} a lui $V$, unde $k\in\mathbb{N}$, poate fi definită 
ca o funcție $c:V\to\{1,2,...,k\}$.
\end{definition}

Colorarea se numește proprie dacă $\forall u,v\in V$, $u$ și $v$ vecine, 
$c(u)\neq c(v)$. 

Graful care conține bucle nu poate avea colorări proprii.

\end{frame}

\begin{frame}
 În cele ce urmează în loc de termenul "colorare proprie" vom 
folosi simplu termenul "colorare".

\end{frame}

\begin{frame}
\includegraphics[scale=0.4]{../../resources/diapozitive11/g90.png} 
3-colorare

\pause
\includegraphics[scale=0.4]{../../resources/diapozitive11/g91.png}
4-colorare

\pause
\includegraphics[scale=0.4]{../../resources/diapozitive11/g92.png}
5-colorare

{\tiny http://www.personal.kent.edu/~rmuhamma/GraphTheory/MyGraphTheory/coloring.htm}
\end{frame}


\begin{frame}
  \frametitle{Numărul cromatic}

Numărul minim de culori necesare unei colorări proprii a unui graf $G$ s.n. 
\emph{numărul cromatic} al grafului și se notează cu $\chi(G)$.

\end{frame}

\begin{frame}
  \frametitle{Numărul cromatic. Marginea de sus}

\begin{theorem}
Fie $G$ un garf simplu atunci $\chi(G)\leq \Delta(G)+1$
\end{theorem}
\pause

\begin{theorem}[Brook]
Fie $G$ un graf simplu conex. Dacă $G$ nu este un graf complet și nu este graful ciclu, atunci 
 $\chi(G)\leq \Delta(G)$.
\end{theorem}

 
\end{frame}



\begin{frame}
  \frametitle{Numărarea colorărilor}

Fie $G=(V,E)$ un graf. Notăm prin $P(G,k)$ sau $\pi_k(G)$ numărul de colorări 
posibile cu $k$ culori.
 
\end{frame}

\begin{frame}
 \frametitle{Numărarea colorărilor la $E_n$}

$E_1$\\
$P(G,k)=k$

$E_2$\
$P(G,k)=k\cdot k$

$E_3$\\
$P(G,k)=k\cdot k\cdot k$

...
$E_n$\\
$P(G,k)=\underbrace{k\cdot k\cdot ... \cdot k}_{n} = k^n$

\end{frame}

\begin{frame}
 \frametitle{Numărarea colorărilor la $K_n$}

$K_1$\\
$P(G,k)=k$

$K_2$\\
$P(G,k)=k(k-1)$

$K_3$\\
$P(G,k)=k(k-1)(k-2)$

...
$K_n$\\
$P(G,k)=\underbrace{k(k-1)(k-2)...(k-(n-1))}_{n}$
\end{frame}

\begin{frame}[allowframebreaks]
\frametitle{Polinomul cromatic}
\begin{theorem}
$P(G,k)$ este un polinom. Mai mult, dacă graful $G$ este simplu atunci 
\[
  P(G,k) = P(G-e,k) - P(G/e,k).
\]
\end{theorem}
\end{frame}

\begin{frame}[allowframebreaks]
\frametitle{Demonstrație}
Demonstrăm prin inducție pe $n=||G||$. Pentru $n=0$ avem graful nul pentru 
care $P(E_n,k) = k^n$ (polinom). Deci teorem este verificată.
\newpage
Presupunem că teorema are loc pentru orice graf cu numărul de muchii mai mic 
decît $n$. Fie $G$ un graf cu $n$ muchii, $n\geq 1$. În $G$ alegem o muchie și 
notăm capetele acesteia prin $u$ și $v$. Dacă eliminăm muchia $e$ obținem două 
grafuri noi: $G-e$ și $G/e$. 
\newpage
Acum, pentru $G-e$ avem mai multe posibilități de colorare decît pentru $G$: 
orice colorare a lui $G-e$ în care $u$ și $v$ au culori diferite este și o 
colorare a lui $G$, iar colorările în care $u$ și $v$ au aceeași culoare nu 
pot fi considerate colorări ale lui $G$.
\newpage
Pe de altă parte aceste colorări sînt colorări pentru $G/e$ întrucît în el 
$u$ și $v$ au fost combinate într-un singur vîrf. Deci
\[
  P(G,k) = P(G-e,k) - P(G/e,k).
\]


\end{frame}

\begin{frame}
Notația $P(G, k)$ se numește polinom cromatic, iar demonstrația teoremei 
poate fi utilizată drept algoritm de determinare a polinomului cormatic.
 
\end{frame}

\begin{frame}
  \frametitle{Proprietăți ale polinomului cromatic}
  

\begin{enumerate}
 \item $P(G,k) = k^n-a_1k^{n-1}+a_2k^{n-2}-...$ unde $a_i\geq 0$ și $n$ este 
 numărul de vîrfuri ale grafului.
\pause
 
 \item $P(G,k) = k^n-mk^{n-1}+...$ unde $m$ este numărul de muchii a grafului $G$
\pause
 \item $P(G,k)$ este divizibil prin $k$
\pause
 \item Semnele coeficienților alternează
\pause
 \item Cel mai mic $i$ pentru care $a_1\neq 0$ reprezintă numărul de componente a grafului
\pause
 \item Dacă $G=G_1\cup G_2$ de grafuri disjuncte, atunci $P(G,k) = P(G_1,k)P(G_2,k)$

\end{enumerate}
 
\end{frame}




\end{document}

