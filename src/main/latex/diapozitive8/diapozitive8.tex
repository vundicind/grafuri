
\title{Curs 8: Grafuri planare}

\begin{document}

\maketitle

\begin{frame}
  \frametitle{Exemplu fabrica; T\' uran}

\end{frame}



\begin{frame}
  \frametitle{Grafuri planar; Graf plan}

Un graf $G$ se numește \emph{planar} dacă poate fi reprezentat în plan astfel încît muchiile să nu se intersecteze decît în vîrfuri. 

O astfel de reprezentare se numește \emph{hartă} (sau \emph{graf-plan}), iar graful $G$ se numește \emph{graful suport} al hărții respective. 

\begin{figure}
\centering%
\begin{tikzpicture}
  \SetVertexNoLabel

  \grComplete[RA=1]{4}
  \grCycle[x=4,RA=1,prefix=v]{4}
  \Edges(v0,v2)
  \Edges[style={bend left,in=90,out=90}](v1,v3)
\end{tikzpicture}
\caption{Graful $K_4$ împreună cu harta sa}
\end{figure}

\end{frame}


\begin{frame}
  \frametitle{Numărul de intersecții}

\emph{Numărul de intersecții} este numărul de perechi diferite de muchii care se intersectează într-o reprezentare a grafului.

Este logic să vorbim despre numărul minim de astfel de perechi.

Un graf planar are numărul minim de intersecții egal cu zero.

\end{frame}

\begin{frame}
  \frametitle{Fețe}

Orice reprezentare a unui graf planar împarte planul în regiuni numite \emph{fețe}. 

Mulțimea fețelor se notează prin $F$. 

Întotdeauna există o față infinită/nemărginită numită \emph{fața exterioară}.

Orice față $f$ este un poligon. 

Numărul de muchii ale poligonului se notează prin $d(f)$ și se numește \emph{gradul} feței.

\begin{theorem}
Pentru orice graf planar
\[
 \sum_{f\in F} d(f) = 2|E|
\]
\end{theorem} 

\end{frame}

\begin{frame}
  \frametitle{Fețe}

\begin{figure}
\centering%
\begin{tikzpicture}
  \SetVertexMath

  \mygrHousePlanar
  \node(f0) at (0,1){$f_0$};
  \node(f1) at (1,0.5){$f_1$};
  \node(f2) at (1.5,-0.3){$f_2$};
\end{tikzpicture}
\caption{$|F|=3$; $f_0$ este fața exterioară}
\end{figure}

\end{frame}

\begin{frame}
  \frametitle{Fețe}

\begin{theorem}
Fie $G$ un graf plan și $e$ o muchie din $G$, atunci:
\begin{enumerate}
  \item Dacă $e$ aparține unui ciclu elementar $C\subseteq G$ atunci $e$ aparține frontierei a exact două fețe;
  \item Dacă $e$ nu aparține unui ciclu elementar atunci $e$ aparține frontierei a exact a unei fețe;
\end{enumerate}

\end{theorem}
\end{frame}

\begin{frame}
  \frametitle{Fețe}

\begin{figure}
\centering%
\begin{tikzpicture}
  \SetVertexMath

  \mygrHousePlanar
  \node(f0) at (0,1){$f_0$};
  \node(f1) at (1,0.5){$f_1$};
  \node(f2) at (1.5,-0.3){$f_2$};
\end{tikzpicture}
\caption{Graf plan $G$}
\end{figure}

Muchia $uv$ aparține ciclului elementar $(u,v,x,u)$ și în același timp aparține frontierei fețelor $f_0$ și $f_1$.

Muchia $xz$ nu aprține unui ciclu elementar și se află pe frontiera feței $f_0$.

\end{frame}

\begin{frame}
  \frametitle{Fețe}

\begin{corollary}
Un arbore plan are exact o față.
\end{corollary}

\begin{corollary}
Dacă un graf plan are fețe diferite cu aceeași frontieră atunci acesta este un graf ciclu.
\end{corollary}

\begin{corollary}
Într-un graf 2-conex orice față este mărginită de un ciclu.
\end{corollary}

\end{frame}

\begin{frame}
  \frametitle{Formula Euler}

 \begin{theorem}
Pentru orice graf $G=(V,E)$ planar și conex
\[
  |V|-|E|+|F|=2
\]
\end{theorem}

\end{frame}



\begin{frame}
  \frametitle{Formula Euler}

 \begin{proof}
Demonstrăm prin inducție tare pe $m=||G||$ (numărul de muchii). 

Pentru $m=0$ avem graful nul cu un singur vîrf (întrucît $G$ trebuie să fie conex); $|V|=1$ și $|F|=1$; teorema este verificată.

Presupunem că teorema este adevărată pentru orice graf cu $|E|<m$. 

Fie $G$ un graf cu $|E|=m$, $m\geq 1$.

Alegem, în mod arbitrar, o muchie $e$ și cercetăm subgraful $H=G-e$. 

Considerăm două cazuri: $H$ este conex și $H$ nu este conex.

\end{proof}

\end{frame}

\begin{frame}
\begin{proof}
{\em Cazul I.} Dacă $H$ este conex, reiese că $e$ nu este o punte în $G$ și deci aparține unui ciclu.

În acest caz $e$ mărginește două fețe diferite, iar în rezultatul eliminării, aceste fețe sau unit în una singură, deci $|F(H)|=|F|-1$. 

Astfel, întrucît $V(G)=V(H)$ obținem:
\[
  \begin{array}{ll}
    |V|-|E|+|F|	&= |V(H)|-(|E(H)|+1)+(|F(H)|+1)\\
		&= |V(H)|-|E(H)|+|F(H)|-1+1 = 2.
  \end{array}
\]
 
\end{proof}

\end{frame}

\begin{frame}
\begin{proof}
{\em Cazul al II-lea.} Graful $H$ nu este conex. 

Reiese că $e$ este o punte și întrucît am eliminat doar o muchie $H$ constă din două compoenente conexe $H_1=(V_1,E_1)$ și 
$H_2=(V_2,E_2)$. 

Iar întrucît atît $H_1$ cît și $H_2$ are un numar de muchii mai mic ca $m$ pentru ele este adevărată formula lui Euler:
\[
 |V_1|-|E_1|+|F_1| = 2
\]
și
\[
 |V_2|-|E_2|+|F_2| = 2
\]
Dar $|V| = |V_1|+|V_2|$, $|E|=|E_1|+|E_2|+1$ și $|F| = |F_1|+|F_2|-1$ și 
\[
 \begin{array}{ll}
   |V|-|E|+|F| &= |V_1|+|V_2|-(|E_1|+|E_2|+1)+(|F_1|+|F_2|-1) \\
   & = |V_1|-|E_1|+|F_1| + |V_2|-|E_2|+|F_2| -1 -1\\
   &= 2 + 2 - 1 -1 =2
 \end{array} 
\]
\end{proof}

\end{frame}

\begin{frame}
  \frametitle{Aplicații ale formulei Euler}

Graful $K_{3,3}$ nu este planar.

\begin{proof}
Presupunem, prin absurd,  că $K_{3,3}$ este planar. Atunci din formula lui Euler 
avem:
\[
 |F| = 2 - |V| + |E| = 2-6+9 = 5.
\]
Deci trebuie să fie $5$ fețe. Deoarece cel mai scurt ciclu în $K_{3,3}$ are 
lungimea 4 reiese că orice față trebuie să aibă gradul $\geq 4$.
Acum
\[
 18 = 2|E| = \sum_{f\in F}d(f) \geq 5\cdot 4 = 20
\]
Contradicție.

\end{proof}
\end{frame}

\begin{frame}
  \frametitle{Aplicații ale formulei Euler}

 $K_5$ nu este planar.


Demonstrați analog această propoziție 
\end{frame}

\begin{frame}
  \frametitle{Aplicații ale formulei Euler}

 Graful Petersen nu este planar.
\end{frame}

\begin{frame}
  \frametitle{Teorema lui Kuratowski}
 

\begin{definition}
O \emph{subdiviziune} a unui graf este un graf nou obținut din $G$ prin 
inserarea de vîrfuri (de gradul 2) în muchiile lui $G$
\end{definition}

Cîteva observații:
\begin{enumerate}
 \item dacă $G$ este planar atunci orice subgraf al acestuia este planar.
 
 \item dacă o subdiviziune a unui graf $G$ este un graf planar atunci $G$ este 
 planar
\end{enumerate}

\begin{theorem}[Kuratowski]
Un graf nu este planar dacă și numai dacă o subdiviziune a lui $K_{3,3}$ sau 
$K_5$ este un subgraf al lui $G$.
\end{theorem}


Arătați că Petersen verifică teorema lui Kuratowski.


\end{frame}


\begin{frame}
  \frametitle{Graf dual}

Fie $G$ un graf plan. 

Amplasăm cîte un vîrf în fiecare față a grafului $G$; și notăm mulțimea acestor vîrfuri prin $V^*$.

Pentru fiecare muchie $e$ unim două vîrfuri din $V^*$ printr-o muchie $e^*$ dacă aceste vîrfuri sînt localizate în fețele cu care acestă muchie este incidentă.

Dacă muchia $e$ este incidentă cu o singură față atunci vîrfului asociat acestei fețe îi facem o buclă care intersectează $e$.

Graful $G^*=(V^*,E^*)$ se numește graful dual al lui $G$. 

\end{frame}

\begin{frame}
  \frametitle{Graf dual}

\begin{figure}
\centering%
\begin{tikzpicture}
  \SetVertexMath
  {
    \SetUpEdge[color=gray!10]
    \tikzset{VertexStyle/.append style={color=gray!10}}
    \renewcommand*{\VertexTextColor}{gray!10}

    \mygrHousePlanar
  }

  \SetVertexNoLabel
  \Vertex[x=2,y=1]{f0}
  \Vertex[x=1,y=0.5]{f1}
  \Vertex[x=1.5,y=-0.3]{f2}
  \Edges(f0,f1) \Edges[style={bend right}](f0,f1)
\end{tikzpicture}
\caption{Graful dual $G^*$}
\end{figure}
\end{frame}

\begin{frame}
  \frametitle{Graf dual}

Tăieturile minimale din $G^*$ sînt cilcurile din $G$ și invers.

\end{frame}


\begin{frame}
 \begin{definition}
Un poliedru este un corp 3 dimensional la care fețele sînt poligoane.
\end{definition}

Un poliedru este convex dacă orice segment care unește două puncte din 
interiorul poliedrului conține numai puncte din interiorul poliedrului. Un 
poliedru convex poate fi proiectat în plan obținînd un graf-plan.

\begin{definition}
Un poliedru convex se numște \emph{corp platonic} dacă există $m\geq 3$ și 
$n\geq 3$ încît orice vîrf are gradul $m$ și orice fața are gradul $n$ (sau 
echivalent, dacă toate fețele sînt poligoane regulate congruente).
\end{definition}

De ce se numesc corpuri platonice?

Cubul este un corp platonic cu $m=3$ și $n=4$.
\end{frame}

\begin{frame}
 \begin{theorem}
Există exact 5 corpuri platonice.
\end{theorem}
\begin{proof}
Fie $G$ un graf planar obținut la proiecția corpului platonic. Atunci 
\[
2|E|=\sum_{v\in V}d(v) = m|V|
\]
\[
2|E|=\sum_{f\in F}d(f) = n|F|
\]
În același timp, din formula lui Euler
\[
 |V|-|E|+|F|=2
\]
În rezultatul, înmulțind cu $mn$
\[
 2mn = mn|V|-mn|E|+mn|F|
 =2n|E|-mn|E|+2m|E|
\]
Deci
\[
 |E|=\frac{2mn}{2n-mn+2m}=\frac{2mn}{4-(m-2)(n-2)}
\]
Acum deoarece $|E|>0$ și $2mn>0$ reise că $(m-2)(n-2)< 4$ și iată toate 
posibilitățile



\end{proof}
\end{frame}

\begin{frame}
 \begin{tabular}{cccccc}
\hline
m&n&|V|&|E|&|F|&platonic solid\\
\hline
3&3&4&6&4&tetraedru\\
4&3&6&12&8&octaedru\\
3&4&8&12&6&cub\\
5&3&12&30&20&icosaedru\\
3&5&20&30&12&dodecaedru
\end{tabular}
\end{frame}


\end{document}

