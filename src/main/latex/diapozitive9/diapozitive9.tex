
\title{Curs 9: Grafuri orientate}

\begin{document}

\maketitle


\begin{frame}
  \frametitle{Graf orientat; Arce}

\begin{definition}
Un \emph{graf orientat} este o pereche $G=(V,E)$ de mulțimi unde $E$ este o mulțime de perechi ordonate de elemente din $V$. 
\end{definition}

Elementele mulțimii $V$ se numesc \emph{vîrfurile} grafului $G$; elementele mulțimii $E$ se numesc \emph{arcele} grafului $G$.

Mulțimea arcelor este o submulțime a produsului cartezian $V\times V$. 

\end{frame}

\begin{frame}
  \frametitle{Reprezentarea grafică}

\begin{figure}
\centering%
\begin{tikzpicture}
  \Vertex{u}
  \NOWE(u){v} \NOEA(u){x} \SOEA(u){y} \SOWE(u){z}
  \SetUpEdge[style={->}]
  \Edge(u)(v) \Edge(u)(x) \Edge(u)(y) \Edge(u)(z)
\end{tikzpicture}
\caption{$G=(\{u,v,x,y,z\},$ $\{(u,v),(u,x),(u,y),(u,z)\})$}
\end{figure}

\end{frame}

\begin{frame}
  \frametitle{Reprezentarea grafică}

\begin{figure}
\centering%
\begin{tikzpicture}
  \Vertex[Lpos=60]{u}
  \NOWE[Lpos=60](u){v} \NOEA[Lpos=60](u){x} \SOEA[Lpos=60](u){y} \SOWE[Lpos=60](u){z}
  \SetUpEdge[style={->}]
  \Edge(v)(z) \Edge(x)(y)
  \SetUpEdge[style={->,bend left}]
  \Edge(v)(x) \Edge(z)(y)
\end{tikzpicture}%
\caption{$H=(V,E)$ unde $V=\{u,v,x,y,z\}$, $E=\{vx,xy,zy,vz\}$}
\end{figure}

\end{frame}

\begin{frame}
  \frametitle{Grade}

\emph{Gradul exterior} al vîrfului $v$ se notează $d^+(v)$ şi este egal cu numărul de arce care au ca extremitate iniţială pe $v$.

\emph{Gradul interior} al vîrfului $v$ se notează $d^-(v)$ şi este egal cu numărul de arce care au ca extremitate finală pe $v$.

Se numește \emph{succesor} al vîrfului $v$ oirce vîrf la care ajunge un arc care iese din vîrful $v$

Se numește \emph{predecesor} al vîrfului vîrf orice vîrf la care intră un arc  in vîrful $v$ 

\end{frame}

\begin{frame}
  \frametitle{Grade}

\begin{figure}
\centering%
\begin{tikzpicture}
  \Vertex[Lpos=60]{u}
  \NOWE[Lpos=60](u){v} \NOEA[Lpos=60](u){x} \SOEA[Lpos=60](u){y} \SOWE[Lpos=60](u){z}
  \SetUpEdge[style={->}]
  \Edge(v)(z) \Edge(x)(y)
  \SetUpEdge[style={->,bend left}]
  \Edge(v)(x) \Edge(z)(y)
\end{tikzpicture}%
\end{figure}

Succesorul vîrfului $x$ este $y$; predecesorul lui $x$ este $v$.

Grade interioare și exterioare: $d^-(x)=d^+(x)=1$; $d^-(y)=2$ și $d^+(y)=0$.

\end{frame}


\begin{frame}
  \frametitle{Drum; Circuit}

Se numeşte \emph{drum} într-un graf orientat o secvenţă de vîrfuri $v_1,v_2,...,v_n$, astfel încît pentru oricare două vîrfuri consecutive $v_i$ şi $v_{i+1}$ există arcul $(v_i,v_{i+1})$. 

Un drum închis, începutul și sfîrșitul coincid, se numește \emph{circuit}. 

\end{frame}

\begin{frame}
  \frametitle{Drum vs Lanț; Circuit vs ciclu}

\begin{figure}
\centering%
\begin{tikzpicture}
  \SetVertexMath

  \SetUpEdge[style={->}]
  \grCycle[RA=1,prefix=v]{4}
  \Vertex[L=v_4,Lpos=180,x=-2,y=0]{v4}
  \Edge(v3)(v1)
  \Edge(v4)(v2)
\end{tikzpicture}%
\end{figure}

Drum de la $v_0$ la $v_3$: $(v_0,v_1,v_2,v_3)$.

Lanț care nu este drum: $(v_1,v_2,v_4)$.

În general nu există drumuri care să se termine în $v_4$.

Cicuit: $v_1,v_2,v_3,v_1$.

Ciclu, dar nu și circuit: $v_0,v_1,v_3,v_0$.

\end{frame}

\begin{frame}
  \frametitle{Secvențe infinite de vîrfuri}

\begin{lemma}
Dacă un digraf conține o secvență infinită de vîrfuri $(v_0,v_1,...)$ astfel încît $v_{i-1}v_i$ este un arc pentru orice $i>0$, atunci $G$ conține un circuit.
\end{lemma}

\begin{figure}
\centering%
\begin{tikzpicture}
  \SetVertexMath
  \tikzset{EdgeStyle/.style = {->}}
  \grWheel[RA=1.5,prefix=v]{8}
\end{tikzpicture}%
\caption{Exemplu de secvența infinită cu prorietatea că $v_{i-1}v_i$ este un arc pentru orice $i$: $(v_0,v_1,...,v_5,v_6,v_0,v_1,...,v_5,v_6,...)$.}
\end{figure}

\end{frame}


\begin{frame}
  \frametitle{Secvențe infinite de vîrfuri}

\begin{proof}
Graful $G$ are un număr finit de vîrfuri; reiese că în secvență sînt vîrfuri care se repetă.

Fie că $v_i=v_j$ pentru un careva $i<j$  și toți $v_k$, $i<k<j$ sînt diferiți (nu se repetă în acest diapazon).

Atunci $v_i,v_{i+1},...,v_j$ este un ciclu în $G$.
\end{proof}

Analog putem demonstra următoarea lemă:

\begin{lemma}
Dacă un digraf conține o secvență infinită de vîrfuri $(v_0,v_1,...)$ astfel încît $v_{i+1}v_i$ este un arc pentru orice $i\geq 0$, atunci $G$ conține un circuit.
\end{lemma}

\end{frame}

\begin{frame}
  \frametitle{Surse și destinații}

\begin{theorem}
Un digraf fără circuite contine cel puţin un vîrf fără succesori şi cel puţin un vîrf fără predecesori.
\end{theorem}


\end{frame}

\begin{frame}
  \frametitle{Surse și destinații}

\begin{proof}
Presupunem că $G$ nu conține vîrfuir fără succesori.

Alegem $v_0$ în baza presupunerii acesta are cel puțin un succesor; alegem unul din ei, de exemplu $v_1$.

Pentru $v_1$ este valabilă aceeași presupunere, deci putem alege succserotul $v_2$ ș.amd.m.d.

Dar atunci lema spune că graful are circuite; Contradicție.
\end{proof}

\end{frame}

\begin{frame}
  \frametitle{Drumuri; Număr cromatic}

\begin{theorem}
Orice graf orientat $G$ conține un drum elementar de lungimea $\chi(G)-1$.
\end{theorem}

\end{frame}

\begin{frame}
  \frametitle{Conexitate tare}

Într-un digraf două vîrfuri se numesc \emph{tare conexe} dacă există un drum de la primul vîrf spre al doilea și invers.

Un digraf este \emph{tare conex} dacă orice două vîrfuri sînt tare conexe.

Vom considera că orice vîrf este tare conex cu el însăși.

Un digraf este \emph{conex} dacă între orice două vîrfuri există un lanț.

Într-un digraf care nu este tare conex putem evidenția \emph{componente tari} [conexe].

\end{frame}


\begin{frame}
  \frametitle{Grafuri orientate remarcabile}

Graf orientat \emph{complet} pe $n$ vîrfuri este graful în care între orice două vîrfuri există un arc.

În general pe $n$ vîrfuri putem construi mai multe grafuri complete.

Un graf orientat se numeşte \emph{antisimetric} dacă pentru oricare două vârfuri din graf $u$ şi $v$ dacă există arcul $(u,v)$, atunci nu există arcul $(v,u)$.

Un graf orientat complet şi antisimetric se numeşte graf \emph{turneu}.
\end{frame}

\begin{frame}
  \frametitle{Digraf asimetric; Turneu}

\begin{figure}
\centering%
\begin{tikzpicture}
  \SetVertexMath

  \SetUpEdge[style={->}]
  \grEmptyCycle[RA=1.5,prefix=v]{4}
  \Edges(v0,v1,v2,v3,v0)
  \Edges(v0,v2) \Edges(v3,v1)

  \grEmptyCycle[x=5,RA=1.5,prefix=v]{4}
  \Edges(v0,v1,v2,v3,v0)
  \Edges(v0,v2) \Edges(v3,v1)
  \Edge[style={->,bend left}](v1)(v0)
\end{tikzpicture}
\end{figure}

Primul digraf este un turneu; al doilea nu este antisimetric și respectiv nu poate fi turneu.

\end{frame}

\begin{frame}
  \frametitle{Grafuri orientate remarcabile}

Notațiile pentru grafurile remarcabile neorientate rămîn aceleași doar că prefixate cu ``D'';

Digraful circuit: $DC_n$.

Digraful drum: $DP_n$.

Digraful complet: $DK_n$.
 
\end{frame}

\begin{frame}
  \frametitle{Orientări}

Un graf neorientat poate fi transformat într-un digraf asociind fiecărei muchii o direcție.

Acest proces se numește \emph{orientare} a grafului.

Dacă un graf neorientat are $m$ muchii acesta poate orientat în $2^m$ moduri; în general printre aceste $2^m$ digrafuri unele sînt izomorfe.

O problemă practică în cazul orientarii grafurilor este de a orienta astfel încît digraful rezultat să fie tare conex.

O astfel de orientare se numește \emph{orientare tare}.

\end{frame}

\begin{frame}
  \frametitle{Orientări tari}

\begin{theorem}
Un graf conex $G$ are o orientare tare dacă și numai dacă orice muchie aparține la cel puțin un ciclu. 
\end{theorem}

\end{frame}

\begin{frame}
  \frametitle{Turnee}

În cazul cînd avem o competiție sportivă în care fiecare participant trebuie să joace cu toți ceilalți participanți și rezultatul fiecărui joc este cîștig sau pierdere; acesta poate fi modelată printr-un digraf.

Participanții reprezentăm prin vîrfuri, iar dacă $x$ a cîștigat jucînd cu $y$ ducem un arc de $xy$.

Pentru $n$ participanți avem un graf orientat complet și asimetric; care se numește turneu.

Gradul exterior al unui vîrf este numărul de cîștiguri al acestui participant.

Dacă avem arcul $xy$ spunem că $x$ \emph{domină} $y$.

Un turneu este reductibil dacă mulțimea vîrfurilor poate fi partiționată în două submulțimi nevide $X$ și $Y$ astfel încît fiecare vîrf din $X$ domină fiecare vîrf din $Y$.

\end{frame}

\begin{frame}
  \frametitle{Turnee}

\begin{theorem}
Un turneu este ireductibil dacă și numai dacă este tare conex. 
\end{theorem}

\end{frame}

\begin{frame}
  \frametitle{Turnee}
 
\begin{proof}
Presupunem că $G$ este un turneu reductibil și fiecare vîrf din $X$ domină fiecare vîrf din $Y$.

Atunci nici un vîrf din $X$ nu este accesibil din $Y$ (nu există drum din $Y$ spre $X$), deci $G$ nu este tare conex.

Presupunem că $G$ nu este tare conex; fie două vîrfuri $u$ și $v$ astfel încît $v$ nu este accesibil din $u$.

Notăm prin $X$ mulțimea tuturor vîrfurilor care nu-s accesibile din $u$.

Și prin $Y$ mulțimea vîrfurilor care-s accesibile din $u$.

Atunci $X$ și $Y$ nu sînt vide deoarece $u\in X$ și $v\in Y$; $X\cap Y=\emptyset$ și $X\cup Y=V(G)$.

În plus orice vîrf din $X$ domină orice vîrf din $Y$. Deci $G$ este reductibil.

\end{proof}
\end{frame}

\begin{frame}
  \frametitle{Turnee}

\begin{corollary}
Orice vîrf dintr-un turneu ireductibil aparține unui circuit de lungimea 3. 
\end{corollary}

\end{frame}

\begin{frame}
  \frametitle{Turnee; Drum Hamilton}

Un drum Hamilton este drum elementar care conține toate vîrfurile digrafului

\begin{theorem}
Orice turneu conține un drum Hamilton. 
\end{theorem}

\end{frame}


\begin{frame}
  \frametitle{Turnee; Drum Hamilton}

\begin{proof}
Fie dat un turneu pe $n$ vîrfuri; adică un $DK_n$.

În baza unei teoreme anterioare în $DK_n$ există un drum elementar de lungimea $\chi(DK_n)-1$.

Acesta este drumul Hamilton căutat deoarece $\chi(DK_n)=n-1$.
\end{proof}


\end{frame}

\begin{frame}
  \frametitle{Turnee; Circuit Hamilton}

\begin{theorem}
Orice vîrf într-un turneu tare conex pe $n$ vîrfuri, $n\geq 3$, aparține unui circuit de lungimea $k$ pentru orice $3\leq k\leq n$.
\end{theorem}

\begin{corollary}
Orice turneu tare conex conține un circuit Hamilton.
\end{corollary}

\end{frame}

\end{document}

