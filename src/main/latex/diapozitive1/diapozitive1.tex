
\title{Curs 1: Grafuri; Introducere}
\date{Bălți, 2013}

\begin{document}

\maketitle

\begin{frame}
  \frametitle{Graf; Vîrfuri; Muchii}

\begin{definition}
Un \emph{graf} este o pereche $G=(V,E)$ de mulțimi unde $E$ este o mulțime de perechi neordonate de elemente din $V$. 
\end{definition}

Elementele mulțimii $V$ se numesc \emph{vîrfurile} grafului $G$; elementele mulțimii $E$ se numesc \emph{muchiile} grafului $G$.

Dacă $e=\{u,v\}$ este o muchie a grafului atunci spunem că $e$ este \emph{incidentă} cu vîrfurile $u$ și $v$; iar $u$ și $v$ sînt \emph{adiacente} (sau \emph{vecine}).

Vîrfurile cu care o muchie este incidentă se numesc \emph{extremitățile} acesteia.

\end{frame}

\begin{frame}
  \frametitle{Reprezentarea grafică}

\begin{figure}
\centering%
\begin{tikzpicture}
  \Vertex{u}
  \NOWE(u){v} \NOEA(u){x} \SOEA(u){y} \SOWE(u){z}
  \Edges(u,v,u,x,u,y,u,z)
\end{tikzpicture}
\caption{$G=(\{u,v,x,y,z\},$ $\{\{u,v\},\{u,x\},\{u,y\},\{u,z\}\})$}
\end{figure}

\end{frame}

\begin{frame}
  \frametitle{Reprezentarea grafică}

\begin{figure}
\centering%
\begin{tikzpicture}
  \Vertex[Lpos=60]{u}
  \NOWE[Lpos=60](u){v} \NOEA[Lpos=60](u){x} \SOEA[Lpos=60](u){y} \SOWE[Lpos=60](u){z}
  \Edge(v)(z) \Edge(x)(y)
  \tikzset{EdgeStyle/.append style = {bend left}}
  \Edge(v)(x) \Edge(z)(y)
\end{tikzpicture}%
\caption{$H=(V,E)$ unde $V=\{u,v,x,y,z\}$, $E=\{vx,xy,yz,zv\}$}
\end{figure}

\end{frame}

\begin{frame}
  \frametitle{Graf vid; Graf trivial; Graf nul}

Graful $(\emptyset,\emptyset)$ se notează simplu prin $\emptyset$ și se numește \emph{graful vid}.

Graful fără vîrfuri sau doar cu 1 vîrf se numește \emph{graf trivial}.

Graful cu 0 muchii se numeste \emph{graf nul} și se notează $N_n$ unde $n\in\mathbb{N}$ este numărul de vîrfuri.

\end{frame}

\begin{frame}
  \frametitle{Numărul de vîrfuri; Numărul de muchii}

Numărul de vîrfuri ale unui graf $G$ se numește \emph{ordinul} grafului $G$; se notează $|G|$. 

Numărul de muchii ale unui graf $G$ se notează $||G||$.

Dacă $|G|=n$ și $||G||=m$, atunci spunem că avem un $(n,m)$-graf.

Pentru a indica faptul că un graf are ordinul $n$ se poate folosi expresia: ``graf pe $n$ vîrfuri''.

\end{frame}


\begin{frame}
  \frametitle{Mulțimea vîrfurilor; Mulțimea muchiilor}

Fiind dat un graf $G$ putem folosi notația $V(G)$ pentru a ne referi la mulțimea de vîrfuri și $E(G)$ a ne referi la mulțimea de muchii.
\begin{itemize}
  \item \alert{De exemplu:} Dacă $G=(\{a,b,c\},\{ab,ac\})$ atunci $V(G)=\{a,b,c\}$, iar $E(G)=\{ab,ac\}$;
  \item \alert{De exemplu:} $V(\emptyset)=\emptyset$ și $E(\emptyset)=\emptyset$.
\end{itemize}

Pentru a indica faptul că un graf are mulțimea vîrfurilor $V$ se poate folosi expresia: ``graf pe $V$''.
\end{frame}

\begin{frame}
  \frametitle{Multigraf}

\begin{figure}
\centering%
\begin{tikzpicture}[scale=1.4]
  \mygrHouseM
\end{tikzpicture}
\end{figure}

\end{frame}

\begin{frame}
  \frametitle{Multigraf}

\begin{definition}
Un \emph{multigraf} este o un triplet $G=(V,E,f)$ care constă  din
două mulțimi disjuncte $V$, $E$ și o \emph{funcție de incidență} $f:E\to V\cup [V]^2$.
\end{definition}

Prin $[V]^2$ am notat mulțimea tuturor perechilor neordonate de elemente din $V$.

Mulțimile $V$ și $E$ sînt multimile de vîrfuri și muchii;

Funcția $f$ pune în corespondență fiecărei muchii capetele acesteia;

Muchiile $e_1, e_2, ...,e_n$ pentru care $f(e_1)=...=f(e_n)$ se numesc \emph{muchii multiple} (sau \emph{paralele});

Iar muchiile pentru care $f$ este un doar un vîrf, $f(e)=\{v\}$ se numesc \emph{bucle}.

\end{frame}

\begin{frame}
  \frametitle{Multigraf}

\begin{definition}
Un graf este o pereche $G=(X,\Gamma)$ formată de mulțimea $X$ și aplicația $\Gamma:X\to X$. 
\end{definition}

\begin{definition}
Un graf este o pereche $G=(X,U)$; unde $X$ este mulțimea vîrfurilor, iar $U\subseteq X\times X$ mulțimea arcelor.
\end{definition}

\end{frame}


\begin{frame}
  \frametitle{Grafuri izomorfe}

\begin{definition}
Două grafuri $G$ și $H$ sînt \emph{izomorfe} dacă există o bijecție $f:V(G)\to V(H)$ cu proprietatea că două vîrfuri $u$ și $v$ sînt adiacente în $G$ dacă și numai dacă $f(u)$ și $f(v)$ sînt adiacente în $H$ pentru orice $u$ și $v$ din $V(G)$. 
\end{definition}

Pentru grafurile izmorfe se utilizează notația $G\sim H$.

O asemenea funcție $f$ se numește \emph{izomorfism} dacă $G \neq H$ și \emph{automorfism} în caz contrar.

Din punct de vedere vizual, grafurile $G$ și $H$ sînt izomorfe dacă pot fi aranjate astfel încît înfățișarea lor să fie identică (desigur, fără a schimba adiacența).

\end{frame}

\begin{frame}
  \frametitle{Grafuri izomorfe}

\begin{figure}
\centering%
\begin{tikzpicture}[scale=1.4]
  \mygrHouse

  \begin{scope}[shift={(4,-1)}]
    \SetGraphUnit{1.2}
    \Vertex[Lpos=180]{a}
    \EA(a){b}
    \NO(b){c}
    \WE[Lpos=180](c){d}
    \SetGraphUnit{1}
    \NOEA(c){e}
    \Edges(a,b,c,d,a)
    \Edges(c,e,a)
  \end{scope}

\end{tikzpicture}
\caption{Grafuri izomorfe}
\end{figure}

\begin{table}
\begin{tabular}{|c|c|}
\hline
$u$ & $b$\\
\hline
$v$ & $a$\\
\hline
$x$ & $c$\\
\hline
$y$ & $d$\\
\hline
$z$ & $e$\\
\hline
\end{tabular}
\caption{Corespondențele}
\end{table}

\end{frame}


\begin{frame}
  \frametitle{Grade [ale vîrfurilor]}

\emph{Gradul} (sau \emph{valența}) unui vîrf $v$ este numărul muchiilor incidente cu $v$ și se noteaza cu $d(v)$. 

Pentru un orice graf $G$ notăm $\delta(G)=min\{d(v): v\in V(G)\}$ și $\Delta(G)=max\{d(v): v\in V(G)\}$. 

Dacă $\delta(G)=\Delta(G)$ atunci graful $G$ se numește \emph{regulat}. 

Dacă $\delta(G)=\Delta(G)=k$ atunci graful $G$ se numește \emph{$k$-regulat}.

\begin{table}
\centering%
\begin{tabular}{|l|l|}
\hline
\textbf{$k$}	& \textbf{Denumire}\\
\hline
0		& graf nul\\
\hline
2		& graf bivalent\\
\hline
3		& graf cubic (sau graf trivalent)\\
\hline
\end{tabular}%
\caption{Grafuri $k$-regulate remarcabile}
\end{table}
  
\end{frame}

\begin{frame}
  \frametitle{Grafuri $k$-regulate}

\begin{figure}
\centering%
\begin{tikzpicture}
  \SetVertexNoLabel
  \SetVertexMath
  \grEmptyCycle[RA=1]{5}

  \grComplete[RA=1,x=4]{3}

  \grComplete[RA=1,x=8]{4}
\end{tikzpicture}
\caption{Grafuri regulate (de la stînga spre dreapta): 0-regulat, 2-regulat, 3-regulat}
\end{figure}

\end{frame}

\begin{frame}
  \frametitle{Cazuri particulare}

Cîte grafuri 1-regulate neizomorfe există?

Un vîrf cu gradul 1 se numește \emph{terminal}.

Un vîrf cu gradul 0 se numește \emph{izolat}.

O buclă mărește gradul vîrfului cu care este incidentă cu 2.

\end{frame}

\begin{frame}
  \frametitle{Cazuri particulare}

\begin{figure}
\centering%
\begin{tikzpicture}
  \SetVertexNoLabel
  \SetVertexMath
  \grCycle[RA=1]{2}

  \begin{scope}[shift={(4,0)}]
    \SetVertexLabel
    \grCycle[RA=1,prefix=u]{4}
    \Vertex{v}   
  \end{scope}
\end{tikzpicture}
\caption{De la stînga spre dreapta: graf 1-regulat, graf cu un vîrf izolat}
\end{figure}

\end{frame}

\begin{frame}
  \frametitle{Proprietăți}

\begin{theorem}
Într-un graf simplu și netrivial există cel puțin două vîrfuri cu același grad. 
\end{theorem}

\begin{theorem}
În orice graf $G$ suma gradelor vîrfurilor este de două ori numărul de muchii, adică 
\begin{equation}
 \sum_{v\in V(G)}d(v)=2|E(G)|.
\end{equation}
\end{theorem}

\begin{corollary}
În orice graf, numărul vârfurilor de grad impar este par.
\end{corollary}
\end{frame}

\begin{frame}
  \frametitle{Secvențe de grade}

O secvență nevidă $(d_1,d_2,...,d_n)$ de numere naturale se numește \emph{secvență grafică} dacă există un graf pe $n$ vîrfuri a cărui grade sînt membrii acestei secvențe.

Suma gradelor dintr-o secvență grafică este un număr par.

Graful pe $n$ vîrfuri a cărui grade sînt membrii secvenței $(d_1,d_2,...,d_n)$ se numește \emph{realizarea} acestei secvențe.

\end{frame}

\begin{frame}
  \frametitle{Secvențe de grade}

\begin{theorem}[Havel-Hakimi]
O secvență descresătoare 
\begin{equation}\label{havel-hakimi:d}
  (d_1,d_2,...,d_n)
\end{equation}
de numere naturale, $d_1\geq 1$ și $n\geq 2$, este secvența de grade a unui graf simplu dacă și numai dacă 
\begin{equation}\label{havel-hakimi:d'}
  (d_2-1,d_3-1,...,d_{d_1+1}-1,d_{d_1+2},...,d_n)
\end{equation}
este secvența de grade a unui graf simplu.
\end{theorem}

Secvența \eqref{havel-hakimi:d'} se obține din \eqref{havel-hakimi:d} prin înlăturarea primului număr și decrementarea următoarelor $d_1$ numere.

\end{frame}

\begin{frame}
  \frametitle{Aplicații}

Teorema Havel-Hakimi poate fi utilizată pentru a determina dacă o secvență de numere naturale reprezintă secvența de grade a unui graf simplu.

De exemplu:

\begin{minipage}{0.49\textwidth}
\[
  \begin{array}{c}
    (4,3,3,3,1)\\
    \downarrow\\
    (2,2,2,0)\\
    \downarrow\\
    (1,1,0)\\
    \downarrow\\
    (0,0)
  \end{array}
\]
Ultima secvență este secvența graful $N_2$ care este simplu.
\end{minipage}
%
\begin{minipage}{0.49\textwidth}
\begin{figure}
\centering%
\begin{tikzpicture}
  \SetVertexNoLabel
  \SetVertexMath
  \grComplete[RA=1]{4}
  \Vertex[x=-2.2,y=0]{v}
  \Edge(a3)(v)
\end{tikzpicture}
\end{figure}
\end{minipage}

\end{frame}

\begin{frame}
  \frametitle{Aplicații}

\begin{minipage}{0.49\textwidth}
\[
  \begin{array}{c}
    (2,2,1,1)\\
    \downarrow\\
    (1,0,1)\\
    \downarrow\\
    (-1,1)
  \end{array}
\]
Ultima secvență nici nu este grafică.
\end{minipage}
%
\begin{minipage}{0.49\textwidth}
\begin{figure}
\centering%
\begin{tikzpicture}
  \SetVertexNoLabel
  \SetVertexMath
  \grEmptyCycle[RA=1]{4}
  \Edges(a0,a2)
  \Loop[style={-},dir=NO,dist=1cm](a1)
  \Loop[style={-},dir=SO,dist=1cm](a3)
\end{tikzpicture}
\end{figure}
\end{minipage}

\end{frame}

\begin{frame}
  \frametitle{Lanțuri [în grafuri]}
 
Un \emph{lanț} este o secvență de vîrfuri și muchii
\[
 (v_0,e_1,v_1,e_2,v_2, ..., v_{n-1},e_n,v_n)
\]
ale unui graf $G$, cu proprietatea că oricare două vîrfuri consecutive din 
lanț $v_{i-1}$ și $v_i$ sînt unite prin muchia $e_i$, $\forall i=\overline{1,n}$.

Vîrfurile $e_1,e_2,..., e_{n-1}$ se numesc \emph{vîrfuri interioare} ale lanțului, iar $v_0$ și $v_n$ - \emph{extremități}.

Dacă lanţul conţine numai muchii distincte atunci se numește \emph{lanţ simplu}.

Dacă lanţul conţine numai vîrfuri distincte atunci el se numește \emph{lanț elementar}.

\end{frame}

\begin{frame}
  \frametitle{Lanțuri}

\begin{figure}
\centering%
\begin{tikzpicture}
  \SetVertexMath
  \Vertices{circle}{v_4,v_1,v_2,v_3}
    \Edges(v_4,v_1,v_2,v_3,v_4)
  \begin{scope}[shift={(4,0)}]
    \Vertices{circle}{v_7,v_8,v_5,v_6}
      \Edges(v_7,v_8,v_5,v_6,v_7)
  \end{scope}
  \Edge[style={bend right}](v_4)(v_5) 
\end{tikzpicture}
\end{figure}

Lanț: $(v_3,v_3v_4,v_4,v_4v_5,v_5,v_5v_8,v_8)$;

Lanț neelementar: $(v_1,v_1v_4,v_4,v_4v_5,v_5,v_5v_8,v_8,v_8v_7,v_7,v_7v_6,v_6,v_6v_5,v_5v_4,v_4,v_4v_3,v_3)$;

\end{frame}

\begin{frame}
  \frametitle{Lanțuri}

Lanțul se poate defini și cu ajutorul muchiilor sale 
\[
  (v_0v_1, v_1v_2, ..., v_{n-1},v_n),
\]
iar în cazul cînd graful $G$ este simplu putem definit lanțul doar cu ajutorul 
vîrfurilor sale
\[
  (v_0,v_1,v_2, ..., v_{n-1},v_n).
\]

De ce în cazul grafului simplu lanțul poate fi definit doar utilizînd vîrfurile sale?

Numărul de muchii din lanț se numeste \emph{lungimea} lanțului.
\end{frame}

\begin{frame}
  \frametitle{Cicluri}

Un lanț în care extremitățile reprezintă același vîrf numește \emph{ciclu}. 

Ciclul este \emph{elementar} dacă vîrfurile interioare sînt distincte. 

O muchie care unește două vîrfuri ale unui ciclu însă nu aparține acestuia se numește \emph{coardă}.

\end{frame}

\begin{frame}
  \frametitle{Cicluri}

\begin{figure}
\centering%
\begin{tikzpicture}
  \SetVertexMath
  \mygrComet
\end{tikzpicture}
\end{figure}

Ciclu: $u_0,v_1,v_0,v_3,u_0$;

Ciclu: $v_0,v_1,v_2,v_3,v_0$;

Ciclu neelementar: $u_0,v_1,v_2,v_3,v_0,v_1,u_0$.

\end{frame}


\begin{frame}
  \frametitle{Grafuri bipartite}

\begin{definition}
Un graf \emph{bipartit} este un graf $G$ cu proprietățile:
\begin{itemize}
  \item există submulțimile $X,Y\subseteq V(G)$ cu $X\cap Y=\emptyset$ și $X\cup Y=V(G)$;
  \item orice muchie are un capăt în $X$ și altul în $Y$.   
\end{itemize}
\end{definition}

Perechea $\{X,Y\}$ se numește \emph{bipartiția} grafului $G$.

\end{frame}

\begin{frame}
  \frametitle{Grafuri bipartite}

\begin{figure}
\centering%
\begin{tikzpicture}
  \SetVertexMath
  \begin{scope}[rotate=90]
    \grEmptyPath[RA=1,RS=0,prefix=u]{3}  
    \grEmptyPath[RA=1,RS=1,prefix=v]{2}    
      \Edge(v1)(u0); \Edge(v0)(u1); \Edge(v1)(u2)
  \end{scope}    
    \draw (0,-1) node (G1){$G_1$};
    
  \begin{scope}[shift={(4,0)},rotate=90]
    \grEmptyPath[RA=1,RS=0,prefix=u]{3}  
    \grEmptyPath[RA=1,RS=1,prefix=v]{3}    
      \Edge(v1)(u0);
  \end{scope}    
    \draw (4,-1) node (G2){$G_2$};
  
  \begin{scope}[shift={(8,1)}]
    \grStar[RA=1.2,prefix=v]{6}  
  \end{scope}  
    \draw (8,-1) node (G3){$G_3$};
\end{tikzpicture}
\end{figure}  

Care sînt bipartițiile grafurilor $G_1$,$G_2$ și $G_3$?

\end{frame}

\begin{frame}
  \frametitle{Grafuri bipartite; Cicluri}

\begin{theorem}
Un graf este bipartit dacă și numai dacă nu conține cilcuri impare. 
\end{theorem}


\end{frame}

\begin{frame}
  \frametitle{Graf conex}

Un graf este \emph{conex} dacă între oricare două vârfuri există un lanț.

Un lanț care unește vîrfurile $u$ și $v$ se numește $u-v$-lanț.

\end{frame}

\begin{frame}
  \frametitle{Graf conex}

\begin{figure}
\centering%
\begin{tikzpicture}
  \SetVertexNoLabel
  \mygrComet
\end{tikzpicture}
\caption{Un graf conex}
\end{figure}

\begin{figure}
\centering%
\begin{tikzpicture}
  \SetVertexNoLabel
  \grPath[RA=1,prefix=a]{2}
  \grCycle[RA=1,x=3,prefix=b]{4}
\end{tikzpicture}
\caption{Un graf neconex}
\end{figure}

\end{frame}

\begin{frame}
  \frametitle{Centru; Rază; Diametru}

\emph{Distanța} dintre două vîrfuri $u,v$ ale unui graf \alert{conex} este numărul minim de muchii ale unui lanţ de la $u$ la $v$; se notează $d(u,v)$.

\emph{Excentricitatea} unui vîrf $v$ este distanţa maximă de la acest vîrf la celelalte vîrfuri; se notează $\varepsilon(v)$

Excentricitatea minimă a vîrfurilor se numește raza grafului $G$; se notează $rad (G)$.

Vîrfurile cu excentricitatea minimă se numesc \emph{centrale}.

\emph{Centrul} grafului este mulțimea tuturor vîrfurilor centrale.

Excentricitatea maximă a vîrfurilor se numește diametrul grafului $G$; se notează $diam (G)$.

\end{frame}

\begin{frame}
  \frametitle{Centru; Rază; Diametru}

\begin{figure}
\centering%
\begin{tikzpicture}
  \SetVertexMath
  \grCompleteBipartite[RA=1,RB=1.5,RS=1.5]{1}{5}
\end{tikzpicture}%
\caption{%
  Un graf $G$;\\
  $\varepsilon(a_0)=1$, $\varepsilon(b_0)=\varepsilon(b_1)=...=\varepsilon(b_4)=2$;\\
  $rad (G)=1, diam (G) = 2$ și unicul vîrf central este $a_0$.%
}
\end{figure}

\begin{figure}
\centering%
\begin{tikzpicture}
  \SetVertexMath
  \mygrHouse
\end{tikzpicture}
\caption{???}
\end{figure}
 
\end{frame}

\begin{frame}
  \frametitle{Proprietăți}

\begin{theorem}
  Pentru orice graf $G$, $rad(G)\leq diam (G)\leq 2 rad (G)$.
\end{theorem}

\end{frame}


\begin{frame}
  \frametitle{Grafuri remarcabile; Graf nul vs. graf complet}

\begin{figure}
\centering%
\begin{tikzpicture}
  \SetVertexNoLabel
  \foreach \n/\x in {3/0,4/4,5/8}
    \grEmptyCycle[RA=1,x=\x]{\n}
    \draw (\x,-1.7) node (N\n){$N_{\n}$}; 
\end{tikzpicture}
\end{figure}

\begin{figure}
\centering%
\begin{tikzpicture}
  \SetVertexNoLabel
  \foreach \n/\x in {3/0,4/4,5/8}
    \grComplete[RA=1,x=\x]{\n}
    \draw (\x,-1.7) node (K\n){$K_{\n}$}; 
\end{tikzpicture}
\end{figure}

\end{frame}

\begin{frame}
  \frametitle{Grafuri remarcabile; Graf nul vs. graf complet}

\begin{definition}[Graf nul]
Un \emph{graf nul} este un graf în totalitate fără muchii, adică de forma $(V,\emptyset)$; 
un graf nul pe $n$ vîrfuri se notează $N_n$, $n\geq 1$.
\end{definition}

\begin{definition}[Graf complet]
Un graf \emph{graf complet} este un graf în care orice 2 vîrfuri diferite sînt adiacente; 
se notează $K_n$, unde $n$, $n\geq 1$, semnifică numarul de vîrfuri ale grafului.
\end{definition}

\end{frame}

\begin{frame}
  \frametitle{Grafuri remarcabile; Graf bipartit vs. graf bipartit complet}

\begin{figure}
\centering%
\begin{tikzpicture}
  \SetVertexNoLabel
  \foreach \p/\q/\x in {2/3/0,4/4/4,1/3/8} {
    \begin{scope}[shift={(\x,0)},rotate=-90]
      \grEmptyPath[RA=1,RS=0,prefix=u]{\p}  
      \grEmptyPath[RA=1,RS=1.5,prefix=v]{\q}    
      \Edge(u0)(v1); \Edge(u0)(v2)
    \end{scope};
    \draw (\x-0.5,-1.7) node (G\p\q){$G_\x$}; 
  }
\end{tikzpicture}
\end{figure}

\begin{figure}
\centering%
\begin{tikzpicture}
  \SetVertexNoLabel
  \foreach \p/\q/\x in {2/3/0,4/4/4,1/3/8} {
    \begin{scope}[shift={(\x,0)},rotate=-90]
      \grCompleteBipartite[RA=1,RB=1, RS=1.5]{\p}{\q}
    \end{scope};
    \draw (\x-0.5,-1.7) node (K\p\q){$K_{\p,\q}$}; 
  }
\end{tikzpicture}
\end{figure}

\end{frame}

\begin{frame}
  \frametitle{Grafuri remarcabile; Graf bipartit vs. graf bipartit complet}

\begin{definition}[Graf bipartit]

\end{definition}

\begin{definition}[Graf bipartit complet]
$K_{p,q}$, $p,q\geq 1$.
\end{definition}

\end{frame}


\begin{frame}
  \frametitle{Grafuri remarcabile; Graf lanț vs. graf ciclu}

\begin{figure}
\centering%
\begin{tikzpicture}
  \SetVertexNoLabel
  \foreach \n/\x in {3/0,4/6}
    \grEmptyPath[RA=1,x=\x]{\n}
    \EdgeInGraphLoop*{a}{\n}
    \draw (0.5*\n+\x,-1) node (P\n){$P_{\n}$}; 
\end{tikzpicture}
\end{figure}

\begin{figure}
\centering%
\begin{tikzpicture}
  \SetVertexNoLabel
  \foreach \n/\x in {3/0,4/4,5/8}
    \grCycle[RA=1,x=\x]{\n}
    \draw (\x,-1.7) node (C\n){$C_{\n}$}; 
\end{tikzpicture}
\end{figure}


\end{frame}

\begin{frame}
  \frametitle{Grafuri remarcabile; Graf lanț vs. graf ciclu}

\begin{definition}[Graf lanț]
Un graf pe $n$ vîrfuri, $n\geq 1$, se numește \emph{graf lanț} dacă constă dintr-un lanț elementar; se noteză $P_n$. 
\end{definition}

\begin{definition}[Graf ciclu]
Un graf pe $n$ vîrfuri, $n\geq 3$, se numește \emph{graf ciclu} dacă constă dintr-un cilcu elementar; se noteză $C_n$. 
\end{definition}

\end{frame}

\begin{frame}
  \frametitle{Grafuri remarcabile; Graf stea vs. graf roată}

\begin{figure}
\centering%
\begin{tikzpicture}
  \SetVertexNoLabel
  \foreach \n/\x/\p in {4/0/a,5/4/b,6/8/c} {
    \begin{scope}[shift={(\x,0)}]
      \grStar[RA=1]{\n}
      \draw (0,-1.7) node (S\n){$S_{\n}$}; 
    \end{scope}
  }
\end{tikzpicture}
\end{figure}

\begin{figure}
\centering%
\begin{tikzpicture}
  \SetVertexNoLabel
  \foreach \n/\x/\p in {4/0/a,5/4/b,6/8/c} {
    \begin{scope}[shift={(\x,0)}]
      \grWheel[RA=1]{\n}
      \draw (0,-1.7) node (W\n){$W_{\n}$}; 
    \end{scope}
  }
\end{tikzpicture}
\end{figure}


\end{frame}

\begin{frame}
  \frametitle{Grafuri remarcabile; Graf stea vs. graf roată}

\begin{definition}[Graf stea]
Un graf pe $n$ vîrfuri, $n\geq 1$, se numește \emph{graf stea} dacă este $K_{1,n-1}$; se noteză $S_n$. 
\end{definition}

\begin{definition}[Graf roată]
Un graf pe $n$ vîrfuri, $n\geq 4$, se numește \emph{graf roată} dacă ...; se noteză $W_n$. 
\end{definition}

\end{frame}

\end{document}

