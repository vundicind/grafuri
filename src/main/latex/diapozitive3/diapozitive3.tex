
\title{Curs 3: Grafuri; Subgrafuri; Operații}

\begin{document}

\maketitle

\begin{frame}
  \frametitle{Subgraf}

Un \emph{subgraf} al unui graf $G$ este un graf $H$ astfel încît $V(H)\subseteq V(G)$, $E(H)\subseteq E(G)$ și pentru orice muchie din $E(H)$ capetele 
acesteia sînt în $V(G)$.\pause

Altfel spus, trebuie să avem $E(H)\subseteq [V(H)]^2$ pentru ca $H$ să poată fi numit subgraf al lui $G$.\pause

Pentru a desemna că $H$ este subgraf al lui $G$ utilizăm notația $H\subseteq G$ și putem spune că $H$ se \emph{conține} în $G$ (sau $G$ conține $H$).\pause

Dacă $H\subseteq G$, $H\neq\emptyset$ și $H\neq G$ spunem că $H$ este un \emph{subgraf propriu} al lui $G$.

\end{frame}

\begin{frame}
  \frametitle{Exemple}

\begin{figure}
\centering%
\begin{tikzpicture}
  \SetVertexMath

  \begin{scope}
    \mygrHouse
    \draw (1,-1.7) node (G){$G$};
  \end{scope}

  \pause

  \begin{scope}[shift={(5,0)}]
    {
      \SetUpEdge[color=gray!40]
      \tikzset{VertexStyle/.append style={color=gray!40}}
      \renewcommand*{\VertexTextColor}{gray!40}

      \mygrHouse
    }
    \Vertex[Lpos=180]{v}
    \NOEA(v){u}
    \SOEA(u){x}
    \SO[Lpos=180](v){z}
    \Edges(v,u)
    \Edges(z,x)
    \draw (1,-1.7) node (H){$H=(\{u,v,x,z\},\{uv,xz\})$};
  \end{scope}

  \pause

  \begin{scope}[shift={(0,-4)}]
    {
      \SetUpEdge[color=gray!40]
      \tikzset{VertexStyle/.append style={color=gray!40}}
      \renewcommand*{\VertexTextColor}{gray!40}

      \mygrHouse
    }
    \Vertex[Lpos=180]{v}
    \NOEA(v){u}
    \SOEA(u){x}
    \SO(x){y}
    \SO[Lpos=180](v){z}
    \draw (1,-1.7) node (I){$I=(\{u,v,x,y,z\},\{\})$};
  \end{scope}

  \pause

  \begin{scope}[shift={(5,-4)}]
    {
      \SetUpEdge[color=gray!40]
      \tikzset{VertexStyle/.append style={color=gray!40}}
      \renewcommand*{\VertexTextColor}{gray!40}

      \mygrHouse
    }
    \Vertex[Lpos=180]{v}
    \NOEA(v){u}
    \SOEA(u){x}
    \SO[Lpos=180](v){z}
    \Edges(v,u)
    \Edges(z,x)
    \Edges(v,y)
    \draw (1,-1.7) node (J){$J=(\{u,v,x,z\},\{uv,xz,vy\})$};
  \end{scope}
  
\end{tikzpicture}
\only<5>{\caption{Structura $J$ nu este subgraf al lui $G$}}
\end{figure}

\end{frame}

\begin{frame}
  \frametitle{Cazuri particulare de subgrafuri}

\begin{itemize}
 \item Dacă $H\subseteq G$ și $V(H)=V(G)$, atunci $H$ se numește 
  \emph{subgraf de acoperire}.\pause
 \item Dacă $H\subseteq G$ și $E(H)$ conține toate muchiile $uv\in E(G)$ cu 
  $u,v\in V(H)$, atunci $H$ se numește subgraf \emph{indus}%
  (sau \emph{generat}) de mulțimea $V(H)$ -- și se notează $H=G[V(H)]$.\pause

 \item Dacă $H\subseteq G$ și $V(G)$ constă \emph{numai} din extremitățile 
  muchiilor din $E(H)$ atunci $H$ se numește subgraf \emph{muchie-indus} 
  de $E(H)$ și se notează $G[E(H)]$.
\end{itemize}

\end{frame}

\begin{frame}
  \frametitle{Exemple}

\begin{figure}
\centering%
\begin{tikzpicture}
  \SetVertexMath

  \begin{scope}
    \mygrHouse
    \draw (1,-1.7) node (G){$G$};
  \end{scope}
\pause
  \begin{scope}[shift={(5,0)}]
    {
      \SetUpEdge[color=gray!40]
      \tikzset{VertexStyle/.append style={color=gray!40}}
      \renewcommand*{\VertexTextColor}{gray!40}

      \mygrHouse
    }
    \Vertex[Lpos=180]{v}
    \NOEA(v){u}
    \SOEA(u){x}
    \SO[Lpos=180](v){z}
    \Edges(v,u,x,z)
    \Edges(v,x)
    \draw (1,-1.7) node (H){$H=G[\{u,v,x,z\}]$};
  \end{scope}
\pause
  \begin{scope}[shift={(5,-4)}]
    {
      \SetUpEdge[color=gray!40]
      \tikzset{VertexStyle/.append style={color=gray!40}}
      \renewcommand*{\VertexTextColor}{gray!40}

      \mygrHouse
    }
    \Vertex[Lpos=180]{v}
    \NOEA(v){u}
    \SOEA(u){x}
    \SO(x){y}
    \SO[Lpos=180](v){z}
    \Edges(v,u)
    \Edges(z,x)
    \Edges(v,y)
    \draw (1,-1.7) node (J){$J=G[\{uv,xz,vy\}]$};
  \end{scope}
  
\end{tikzpicture}
\end{figure}

\end{frame}

\begin{frame}
    \frametitle{Cazuri particulare de subgrafuri}

Evident, orice graf $G$ își este subgraf de acoperire.

Evident, $G=G[V(G)]$.

\end{frame}

\begin{frame}
  \frametitle{Maximalitate/minimalitate}

Spunem că un subgraf $H$ este \emph{maximal} în raport cu o proprietate dacă nu există un alt subgraf $I$ cu acestă proprietate și $H\subset I$.\pause

Spunem că un subgraf $H$ este \emph{minimal} în raport cu o proprietate dacă nu există un alt subgraf $I$ cu acestă proprietate și $H\supset I$.

\end{frame}

\begin{frame}
  \frametitle{Maximalitate/minimalitate}

\begin{figure}
\centering%
\begin{tikzpicture}
  \SetVertexMath

  \begin{scope}
    \grCycle[RA=1,prefix=v]{4}
    \Vertex[L=v_4]{v4}
    \Edges(v1,v4)
    \begin{scope}[on background layer]
      \draw[line width=14pt, color=gray!40, rounded corners=7pt] (v0.center) -- (v1.center) -- (v2.center) -- (v3.center) -- (v0.center);
    \end{scope}
    \draw (0,-2) node (G) {$G$};
  \end{scope}

\only<3->{

  \begin{scope}[shift={(4,0)}]
    \grCycle[RA=1,prefix=v]{4}
    \SetUpVertex[Lpos=90]
    \grPath[RA=1,prefix=u,x=2]{2}
    \begin{scope}[on background layer]
      \node[draw,rectangle,rounded corners,fill=gray!40,gray!40,fit=(v0) (v1) (v2) (v3)] {};
      \node[draw,rectangle,rounded corners,fill=gray!40,gray!40,fit=(u0) (u1)] {};
    \end{scope}
    \draw (1.5,-2) node (H) {$H$};
  \end{scope}
}
\end{tikzpicture}
\end{figure}

\only<2->{
În graful $G$ subgraful evidențiat [cu sur] este un subgraf minimal care conține un ciclu.
}

\only<4->{
În graful $H$ cel două subgrafuri evidențiate [cu sur] sînt maximal conexe.
}

\end{frame}

\begin{frame}
  \frametitle{Componente conexe}

Subgrafurile maximal conexe se numesc \emph{componente conexe} (sau simplu \emph{coponente}).\pause

Un graf conex constă dintr-o singură comonentă conexă.

\end{frame}

\begin{frame}
  \frametitle{Componente conexe}

\begin{figure}
\centering%
\begin{tikzpicture}
  \SetVertexMath
  \SetVertexNoLabel

  \begin{scope}
    \grCycle[RA=1,prefix=v]{4}
    \Vertex[L=v_4]{v4}
    \draw (0,-2) node (G) {$G$};
  \end{scope}

  \begin{scope}[shift={(4,0)}]
    \grEmptyCycle[RA=1,prefix=v]{4}
    \Vertex[L=v_4]{v4}
    \Edges(v0,v3)
    \Edges(v1,v2)
    \draw (0,-2) node (H) {$H$};
  \end{scope}

\end{tikzpicture}
\end{figure}

Graful $G$ constă din 2 componente conexe.

Graful $H$ constă din 3 componente conexe.

\end{frame}

\begin{frame}
  \frametitle{Operații; Suprimarea unui vîrf}

\emph{Suprimarea} unui vîrf $v$ dintr-un graf $G$ presupune îndepărtarea vîrfului propriuzis și îndepărtarea tuturor muchiilor incidente cu $v$; se notează $G-v$.\pause

Echivalent, $G-v=G[V(G)\setminus\{v\}]$.\pause

\begin{figure}
\begin{tikzpicture}
  \SetVertexNoLabel
  \grComplete[RA=1,rotation=45]{4}
  \SetVertexLabel
  \Vertex{v}
     \draw (0,-1.5) node (K4){$K_4$}; 

  \draw[->,ultra thick,line width=0.5mm] (1.5,0) -- (2.5,0);     
     
  \SetVertexNoLabel
  \grCycle[RA=1,rotation=45,x=4]{4}  
     \draw (4,-1.5) node (K4v){$K_4-v$};   
\end{tikzpicture}
\end{figure}

Sinonime pentru operația ``suprimare'': ``ștergerea'', ``înlăturarea'', ``îndepărtarea'', ``eliminarea'' unui vîrf.

\end{frame}

\begin{frame}
  \frametitle{Suprimarea unei muchii}

\emph{Suprimarea} unei muchii $e$ dintr-un graf $G$ presupune îndepărtarea doar muchiei propriuzise; se notează $G-e$.\pause

Echivalent, $G-e$ este graful $(V(G), E')$ cu $E'=E\setminus\{e\}$.\pause

\begin{figure}
\centering%
\begin{tikzpicture}
  \Vertices{circle}{v,x,y,z}
  \Edges(v,x,y,z,v)
    \draw (0,-2) node (Gv){$G$};
    
  \draw[->,ultra thick,line width=0.5mm] (2.5,0) -- (3.5,0); 

  \begin{scope}[shift={(6,0)}]
    \Vertices{circle}{v,x,y,z}
    \Edges(v,x) \Edges(y,z,v)
      \draw (0,-2) node (Gxy){$G-xy$};
  \end{scope}  

\end{tikzpicture}
\end{figure}

\end{frame}

\begin{frame}
  \frametitle{Contracția muchiilor}

\emph{Contracția} unei muchii $e=uv$ [într-un vîrf $v_e$] presupune suprimarea muchiei $e$ și înlocuirea vîrfurilor $u,v$ printr-un singur vîrf $v_e$ adiacente vecinilor atît vecinilor lui $u$ cît și vecinilor lui $v$.\pause

Contracția unei muchii $e$ a unui graf $G$ se notează $G/e$.\pause

Formal $G/e=(V(G)\setminus\{u,v\}\cup \{v_e\},E')$ unde $v\notin V(G)\cup E(G)$, iar 
\[
  \begin{array}{ll}
    E'&=\{xy\in E(G):xy\cap uv=\emptyset\}\\
      &\cup \{v_ey:uy\in E(G)\setminus\{e\} \text{ sau } vy\in E(G)\setminus\{e\}\}.
  \end{array}
\]


 
\end{frame}

\begin{frame}
  \frametitle{Contracția muchiilor}

\begin{figure}
\centering%
\begin{tikzpicture}
  \SetVertexMath
  \SetVertexNoLabel
 
  \grCycle[RA=1,prefix=u]{4}
  \grCycle[RA=1,prefix=v,x=3]{4}
  \Edge(u0)(v2) 
  \SetVertexLabel  
  \Vertex[x=1,y=0,Lpos=90]{u}  \Vertex[x=2,y=0,Lpos=90]{v}
    \draw (1.5,-2) node (G){$G$};
    
  \draw[->,ultra thick,line width=0.5mm] (4.5,0) -- (5.5,0);   

  \begin{scope}[shift={(7,0)}]
    \SetVertexNoLabel
    \grCycle[RA=1,prefix=u]{4}
    \grCycle[RA=1,prefix=v,x=2]{4}
    \SetVertexLabel  
    \Vertex[x=1,y=0,Lpos=90,L=v_e]{ve}
      \draw (1,-2) node (Guv){$G/uv$};
  \end{scope}    
\end{tikzpicture}
\end{figure}
  
\end{frame}

\begin{frame}
  \frametitle{Adăugarea unei muchii}

\emph{Suma} dintre un graf $G$ și o muchie $e$ se notează $G+e$ și este graful $(V(G)\cup V(e),E(G)\cup e)$. 

\end{frame}


\begin{frame}
  \frametitle{Generalizarea unor operațiilor}
 
Ștergerea unei mulțimi de vîrfuri $U\subseteq V$,
\[
  G-U=G[V\setminus U].
\]\pause

Ștergerea unei mulțimi de muchii $F\subseteq E$,
\[
  G-F=(V,E\setminus F).
\]\pause

Suma dintre un graf $G$ și o muțime de muchii $F$,
\[
  G+F=(V\cup V(F),E\cup F).
\]


\end{frame}

\begin{frame}
  \frametitle{Maximalitate/minimalitate}

Spunem că un graf $G$ este \emph{muchie-maximal} cu o anumită proprietate dacă $G$ are acestă proprietate, dar $G+uv$, pentru orice vîrfuri neadiacente $u$ și $v$ din $G$, nu are această proprietate.\pause

Spunem că un graf $G$ este \emph{muchie-minimal} cu o anumită proprietate dacă $G$ are acestă proprietate, dar $G-uv$, pentru orice vîrfuri adiacente $u$ și $v$ din $G$, nu are această proprietate.\pause

Un graf $G$ este \emph{maximal} cu o anumită proprietate dacă $G$ are această proprietate, iar orice alt graf $H$ cu $H\supset G$ nu are acestă proprietate.\pause

Un graf $G$ este \emph{minimal} cu o anumită proprietate dacă $G$ are această proprietate, iar orice alt subgraf $H$ cu $H\subset G$ nu are acestă proprietate.

\end{frame}

\begin{frame}
  \frametitle{Maximalitate/minimalitate}

\begin{figure}
\centering%
\begin{tikzpicture}
  \SetVertexMath

  \begin{scope}
    \grEmptyCycle[RA=1,prefix=v]{4}
    \Edges(v0,v1)
    \Edges(v2,v3)
  \end{scope}

  \begin{scope}[shift={(4,0)}]
    \SetUpVertex[Lpos=15]
    \grStar[RA=1,prefix=v]{4}
  \end{scope}

\end{tikzpicture}
\caption{Aceste grafuri sînt muchie-maximale cu $P_3\subseteq G$} [p.165, Diestel]
\end{figure}

 
\end{frame}

\begin{frame}
  \frametitle{Punți}

Fiind dat un graf $G$, o muchie $e\in E(G)$ se numește \emph{punte} dacă $G-e$ are mai multe componenete conexe decît $G$.\pause

Sînt grafuri care nu au punți, de exemplu, $K_n$ sau $C_n$.\pause

Dacă $G$ este conex și orice muchie a sa este punte $\Leftrightarrow$ $G$ este muchie-minimal conex.

\end{frame}

\begin{frame}
  \frametitle{Punți}

\begin{figure}
\centering%
\begin{tikzpicture}
  \SetVertexMath

  \mygrHouse
\end{tikzpicture}
\caption{Muchia $zx$ este punte; celelalte muchii nu sînt punți.}
\end{figure}
 
\end{frame}


\begin{frame}
  \frametitle{Vîrfuri de articulare}

Fiind dat un graf $G$, un vîrf $v\in V(G)$ se numește \emph{vîrf de articulare} dacă $G-v$ are mai multe componenete conexe decît $G$.

\end{frame}

\begin{frame}
  \frametitle{Vîrfuri de articulare}

\begin{figure}
\centering%
\begin{tikzpicture}
  \SetVertexMath

  \SetUpVertex[Lpos=10]
  \grPath[RA=1,prefix=v]{5}
\end{tikzpicture}
\caption{Toate vîrfurile cu excepția vîrfurilor $v_0$ și $v_4$ sînt vîrfuri de articulare}
\end{figure}
 
\end{frame}


\begin{frame}
  \frametitle{Punți; Cicluri}

\begin{theorem}\label{GTA-2.3}
O muchie $e$ a unui graf conex $G$ este punte dacă și numai dacă nu există un 
ciclu în $G$ [în] care să [se] conțină această muchie.
\end{theorem}
\begin{proof}[Demonstrație; Necesitatea]
Fie $e\in E(G)$ o punte în $G$ atunci $G-e$ conține mai multe componente decît $G$. \pause

Adică există cel puțin două vîrfuri $u$ și $v$ care-s conexe în $G$, dar nu și în $G-e$. \pause

Acest fapt implică existența unui $uv$-lanț $P$ care trece prin $e$ (de fapt toate $uv$-lanțurile trec prin $e$). \pause

Notăm prin $x$ și $y$ capetele muchiei $e$ și considerăm că $x$ precede $y$ în $P$. 

\end{proof} 
\end{frame}



\begin{frame}
  \frametitle{Punți; Cicluri}

\begin{proof}[Demonstrație; Necesitatea; Continuare]

Așadar în $G-e$ vîrful $u$ este conectat cu $x$ printr-o secțiune a lui $P$ și $y$ este conectat cu $v$ prin altă secțiune a lui $P$. \pause

Dacă în $G$ ar fi existat un ciclu $C$ care ar conține muchia $e$ atunci $x$ și $y$ ar fi conectați în $G-e$ prin lanțul $C-e$ și respectiv $u$ și $v$ ar fi conectați în $G-e$. \pause

Am obținut o contradicție.
 
\end{proof}


\end{frame}


\begin{frame}
  \frametitle{Punți; Cicluri}

\begin{proof}[Demonstrație; Suficiența]
Vom demonstra implicația inversă, adică dacă $e$ nu este punte atunci ea se conține într-un ciclu. \pause

Presupunem că $e$ nu este punte, atunci $G-e$ are același numar de componente ca și $G$. \pause

Notînd prin $x$ și $y$ capetele muchiei $e$ reiese că $x$ și $y$ sînt conectate în $G-e$.\pause

Adică există un $xy$-lanț $P$ în $G-e$.\pause
 
Atunci $e$ se conține în ciclul $P+e$ din $G$.\pause

Presupunem că teorema este adevărată pentru orice două vîrfuri la distanță mai mică decît $k$. \pause

Fie $u$, $v$ două vîrfuri la distanța $k$. 

\end{proof} 
 
\end{frame}

\begin{frame}

\begin{proof}[Demonstrație; Suficiența; Continuare]

Adică între ele există un $k$-lanț $P$. \pause

Fie $w$ un vîrf din $P$ care precede $v$. \pause

Atunci $d(u,w)=k-1$ și deci între $u$ și $w$ există două lanțuri independente $P$ și $Q$. \pause

Deoarece $G$ este 2-conex reiese că dacă eliminăm $w$ graful rămîne conex și deci între $u$ și $v$ există un lanț $P'$ în $G-w$.\pause

Acum dacă $P'$ nu intersectează nici $P$ nici $Q$ teorema este demonstrată. \pause

Deaceea presupunem fără a pierde din generalitate că $V(P')\cap V(P)=x$ și deci iată lanțurile independente căutate: primul: $u$, secțiunea din $P$ de la $u$ spre $x$, $x$, secțiunea din $P'$ de la $x$ spre $v$, $v$.\pause

Al doilea: $Q$ împreună cu $wv$.
 
\end{proof}

 
\end{frame}





\end{document}

