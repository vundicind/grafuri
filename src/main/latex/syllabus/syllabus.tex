\documentclass[a4paper,11pt]{article}
%twocolumn
%\usepackage[top=0cm,right=0.8cm,bottom=1.2cm,left=0.8cm]{geometry}
%  \setlength{\columnsep}{distance}
%  \setlength{\columnseprule}{thickness}

\usepackage{exercise}
  \renewcommand{\ExerciseName}{}
  \renewcommand{\ExerciseListName}{}
  \renewcommand{\ExerciseHeaderTitle}{\ExerciseTitle}
  \renewcommand{\ExerciseHeader}{\textbf{
                \ExerciseName\ExerciseHeaderNB.\ExerciseHeaderTitle
                \ExerciseHeaderOrigin}}
  \renewcommand{\ExerciseHeader}{\textbf{
                \ExerciseName\ExerciseHeaderNB.\ExerciseHeaderTitle}}
  \renewcommand{\ExerciseListHeader}{\ExerciseHeaderDifficulty%
              \textbf{\ExerciseListName\ \ExerciseHeaderNB.%
              \ \ExerciseHeaderTitle}%
              \ExerciseHeaderOrigin\ignorespaces}
  \renewcommand{\ExerciseListHeader}{\ExerciseHeaderDifficulty%
              \textbf{\ExerciseListName\ \ExerciseHeaderNB.%
              \ \ExerciseHeaderTitle}%
              \ignorespaces}
  \setlength{\ExerciseSkipBefore}{0\baselineskip}
  \setlength{\ExerciseSkipAfter}{0\baselineskip}
  \setlength{\Exesep}{0\baselineskip}
  \setlength{\Exetopsep}{0\baselineskip}
  \setlength{\Exeleftmargin}{25em}
  \setlength{\QuestionBefore}{0\baselineskip}

\usepackage{enumitem}
  \setlist[0]{itemsep=-5pt}
  \setlist{nolistsep}

\usepackage[utf8x]{inputenc}
\usepackage[romanian]{babel}
%\usepackage[T2A]{fontenc}
\usepackage{amsmath}
\usepackage{amsfonts}
\usepackage{amssymb}
\usepackage{hyperref}
\usepackage{ntheorem}
  \theoremstyle{change}
  \theorembodyfont{\upshape}
  \newtheorem{theorem}{Teoremă}[section]
  \newtheorem{definition}[theorem]{Definiție}
  \newtheorem{example}[theorem]{Exemplu}
  \newtheorem{proposition}[theorem]{Propoziție}
  \newtheorem{corollary}[theorem]{Corolar}
  \newenvironment{proof}{{\bf Demonstrație:} }{}
  %\newtheorem{exercise}[theorem]{Exercițiu}
  \newtheorem{question}[theorem]{\^{I}ntrebare}
  \newtheorem{problem}[theorem]{Problemă}
  \newtheorem{algorithm}[theorem]{Algoritm}
\usepackage{verbatim}
\usepackage{graphicx}
\usepackage{hyperref}
\usepackage{color}
\usepackage{caption}
\usepackage{subcaption}
\usepackage{titlesec}
  \titleformat{\chapter}
    {\normalfont\huge\bfseries}{}{20pt}{\Huge}
  \titlespacing*{\chapter}{0pt}{50pt}{40pt}
  \titleclass{\section}{straight}
\usepackage{xargs}
\usepackage{minibox}% http://tex.stackexchange.com/questions/8680/how-can-i-insert-a-newline-in-a-framebox
  \renewcommandx\minibox[3][1=l, 2=c]{%
    \begin{tabular}[#2]{@{}#1@{}}
      #3
    \end{tabular}%
  }
\author{Radu Dumbrăveanu}

\title{Teoria grafurilor\\{\small Syllabus}}

\begin{document}

\maketitle

\section{Informaţii generale despre curs, seminar}

{\bf Titlul disciplinei:} Teoria grafurilor\\
%{\bf Codul:} ???\\
%{\bf Numărul de credite ECTS:} ???\\
{\bf Semestrul:} al II-lea (Anul IV, Ciclul licență)\\
{\bf Ore curs:} 4-2h/săptămînă (24h/semestru)\\
{\bf Ore de seminar:} 2-4h/săptămînă (24h/semestru)\\
{\bf Pagina web a disciplinei:} \url{http://weten.usb.md/moodle2/course/view.php?id=218}

\section{Informaţii despre titularul de curs, seminar}

{\bf Nume, titlul ştiinţific:} Lector universitar Radu Dumbrăveanu\\
{\bf Informaţii de contact:} \url{vundicind@gmail.com}

%\section{Obiectivele disciplinei:}

%Probleme de combinatorică și teoria grafurilor. Ioan Tomescu. Editura didactică și pedagogică. Bucuresti. 1981
%  Arbori; Conexitate; Colorare; Cicluri hamiltoniene.

\section{Modul de evaluare}
Lucrări de evaluare pe parcursul semestrului (???).\\
Examen final (scris).\\
Nota finală se compune din: nota medie (60\%) și nota de la examen (40\%) obținute la acest curs.


\section{Cuprinsul orelor de curs pe parcursul semestrului}

\begin{enumerate}[label=S\v apt\v am\^ ina \arabic*:]
  \setlist[enumerate,2]{label=Curs \arabic*, leftmargin=*}
  \item
    \begin{enumerate}
      \item Grafuri; Introducere
      \item Grafuri Euler și Hamilton
    \end{enumerate}
  \item
    \begin{enumerate}[resume] % http://micheljansen.org/blog/entry/362
      \item Grafuri; Operații
    \end{enumerate}
  \item
    \begin{enumerate}[resume]
      \item Arbori
      \item Conexitate
    \end{enumerate}
  \item
    \begin{enumerate}[resume]
      \item Cuplaje
    \end{enumerate}
  \item
    \begin{enumerate}[resume]
      \item Colorare
      \item Grafuri planare
    \end{enumerate}
  \item
    \begin{enumerate}[resume]
      \item Grafuri orientate
    \end{enumerate}
  \item
    \begin{enumerate}[resume]
      \item Fluxuri
      \item Spațiul ciclurilor
    \end{enumerate}
  \item
    \begin{enumerate}[resume]
      \item Grafuri infinite
    \end{enumerate}
\end{enumerate}

\begin{thebibliography}{9}

\bibitem{a} {\bf Claude Berge, {\em Teoria grafurilor și aplicațiile ei}, Traducere din limba franceză, Editura Tehnică, București, 1969.}

\bibitem{b} {\bf Ioan Tomescu, {\em Probleme de combinatorică și teoria grafurilor}, Editura didactică și pedagogică, Bucuresti, 1981.}

\bibitem{c} Dragoș-Radu Popescu, {\em Combinatorică și teoria grafurilor}, 2005.
%Accesibilă pe Scribd la adresa: \url{http://www.scribd.com/doc/93814356/Dragos-Radu-Popescu-Teoria-Grafurilor}

\bibitem{d} Prof. Popescu Rozica - Maria, {\em Lecţii complementare de teoria grafurilor}.\\
Disponibil pe Internet: \url{http://bibliotecascolara.ro/popescutache/Lectii_complementare_de_teoria_grafurilor.pdf}.
Consultat la 2.01.2013.

\bibitem{e} Professor J.L. Gross, {\em COMS W4203 - GRAPH THEORY}. 
Materialele cursului sînt disponibile pe Internet: \url{http://www.cs.columbia.edu/~cs4203/course_material.html}.
Consultat la 2.01.2013.

\bibitem{f} B\' ela Bollob\' as, {\em Modern Graph Theory}, Springer Verlag, New York, 1998. Graduate Texts in Math-
ematics. Vol 184. ISBN 0-387-98488-7.\\
Accesibilă pe Google Books la adresa: {\footnotesize \url{http://books.google.md/books?id=SbZKSZ-1qrwC}} 

\bibitem{g} R. Diestel, {\em Graph theory}.\\ 
Accesibilă pe pagina web a autorului: \url{http://diestel-graph-theory.com/index.html}.
Consultat la 2.01.2013.

\bibitem{h} J.A. Bondy, U.S.R. Murty, {\em Graph theory with applications}.\\

\end{thebibliography}

Referințele care nu-s cu {\bf bold} sînt opționale.

\end{document}
